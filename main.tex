\documentclass[16pt]{article}
\usepackage[utf8]{inputenc}
\usepackage[russian]{babel}
\usepackage[urlcolor=blue]{hyperref}
\usepackage{amsthm}

\newtheorem{theorem}{Теорема}[section]
\newtheorem{lemma}[theorem]{Лемма}
\theoremstyle{definition}
\newtheorem{definition}[theorem]{Определение}
\newtheorem{example}{Пример}[theorem]

\begin{document}

\href{https://yadi.sk/d/d-ti\_TZi3Mh3Ri/7\%20sem/\%D0\%94\%D0\%93\%D0\%A1\%D0\%9F}{Здесь лежит} учебник Круглова, издание 2016 года.

\newpage

\section{Теорема Серпинского (1.2.7)}
Обозначение класса всех подмножеств: $2^\Omega$.
\begin{definition}
Класс $\mathcal{E} \in 2^\Omega$ называется 
\begin{itemize}
    \item \textit{$\pi$-классом}, если он замкнут относительно пересечения своих элементов ($\forall A, B \in \mathcal{E} : A \cap B \in \mathcal{E}$)
    \item \textit{алгеброй}, если $\Omega, A \cup B, A^c \in \mathcal{E}$
    \item \textit{$\sigma$-алгеброй}, если он является алгеброй и содержит счётное объединение любых своих подмножеств. $\sigma$-алгеброй, \textit{порождённой} $L$, называется минимальная по включению $\sigma$-алгебра, содержащая $L$. Обозначается $\sigma(L)$. 
    \item \textit{монотонным}, если $\forall A_n \uparrow$ : $\cup_{i=1}^\infty A_i \in \mathcal{E}$, $\forall A_n \downarrow$ : $\cap_{i=1}^\infty A_i \in \mathcal{E}$
    \item \textit{$\lambda$-классом}, если $\Omega, A \backslash B, \cup_{i=1}^\infty A_i \in \mathcal{E}$ при $B \subseteq A, A_n \uparrow$.
\end{itemize}
\end{definition}
Если класс является $\pi$- и $\lambda$-классом, то он $\sigma$-алгебра. Кроме того, все эти специальные классы замкнуты относительно пересечения, то есть любое пересечение, например, алгебр является алгеброй. \newline
Если $L$ - один из специальных классов, определённых выше, то $L \cap B$ - класс того же типа, $\forall B \in 2^\Omega$. Кроме того, если $L$ - образ некоторого отображения, то его прообраз также является классом того же типа. Если же есть $L$ - прообраз некоторого отображения в $\Omega'$, то класс множеств из $\Omega'$, прообразы которых лежат в $L$, является классом того же типа.
\begin{theorem}[Серпинский]
Если $\pi$-класс $\mathcal{E}$ содержится в $\lambda$-классе $\mathcal{D}$, то $\sigma(\mathcal{E}) \subseteq \mathcal{D}$. Если алгебра $\mathcal{A}$ содержится в монотонном классе $\mathcal{M}$, то $\sigma(\mathcal{A}) \subseteq \mathcal{M}$. 
\end{theorem}
\begin{proof}[Идея доказательства]

\end{proof}

\section{Измеримое пространство, прямое произведение измеримых пространств, цилиндрические множества (1.2.14 - 1.2.18)}
Обозначение прямого произведения: $\times_{t \in T}\Omega_t$.
\begin{definition}
Пара $(\Omega, \mathcal{F})$, состоящая из некоторого множества $\Omega$ и некоторой $\sigma$-алгебры $\mathcal{F} \subseteq 2^\Omega$, называется \textit{измеримым пространством}. Множества из $\mathcal{F}$ называются \textit{измеримыми множествами}.
\end{definition}

\begin{definition}
Пусть есть множество измеримых пространств $(\Omega_t, \mathcal{F}_t), t \in T$. Тогда прямоугольник $\times_{t \in T}A_t$ называется \textit{измеримым}, если все $A_t \in \mathcal{F}_t$.
\end{definition}

\begin{definition}
Сигма-алгебра, порождённая измеримыми прямоугольниками с $A_t = \Omega_t$ для почти всех $t \in T$, называется \textit{прямым произведением} $\sigma$-алгебр $\mathcal{F}_t$ и обозначается $\otimes_{t \in T}\mathcal{F}_t$
\end{definition}

\begin{definition}
Измеримое пространство $(\times_{t \in T}\Omega_t, \otimes_{t \in T}\mathcal{F}_t)$ называется \textit{прямым произведением} измеримых пространств $(\Omega_t, \mathcal{F}_t), t \in T$.
\end{definition}

\begin{definition}
Для любых $U \subset T, A \subseteq \otimes_{t \in T}\mathcal{F}_t$ множество $C_U(A)$ функций из $(\times_{t \in T}\Omega_t$, сужения которых на $U$ принадлежат $A$, называется \textit{цилиндрическим множеством с основанием $A$}. Если $U$ конечное (счётное), то $C_U(A)$ называется цилиндрическим множеством с \textit{конечномерным} (\textit{счётно-конечным}) основанием.
\end{definition}
В обозначениях выше, справедливы следующие теоремы:
\begin{theorem}
Класс $\mathcal{A}$ цилиндрических множеств с конечномерными основаниями является алгеброй и $\sigma(\mathcal{A}) = \otimes_{t \in T}\mathcal{F}_t$.
\end{theorem}
\begin{proof}[Идея Доказательства]

\end{proof}
\begin{theorem}
Пусть множество $B \subseteq \times_{t \in T}\Omega_t$. \newline $B \in \otimes_{t \in T}\mathcal{F}_t \Longleftrightarrow B = C_U(A)$ с конечномерным или счётным основанием $A$.
\end{theorem}
\begin{proof}[Идея Доказательства]

\end{proof}
\begin{example}
Существуют множества не из прямого произведения сигма-алгебр. Положим $T = [0, 1], \Omega_t = [0, 2], \mathcal{F}_t = \mathcal{B}([0, 2])$. Множество $B = \{\omega \in \times_{t \in T}\Omega_t \mid \sup_{t \in T}\omega_t = 1\}$ не лежит в $\otimes_{t \in T}\mathcal{F}_t$. Доказываем от противного, применяя предыдущую теорему. По ней основание $A$ цилиндрического множества $B$ должно быть прямым произведением счётного числа $\mathcal{F}_{t_n}$. Поскольку мощность $T$ - континуум, можно изменить функцию $\omega \in B$, положив её равной 2 во всех точках $[0, 1] \backslash \{t_n\}$. Тогда её принадлежность $B$ не изменится, но супремум уже будет равен 2, что приводит к противоречию.
\end{example}

\section{Понятие случайного процесса. Теорема Колмогорова о существовании случайного
процесса с данными конечномерными распределениями (3.1.1 – 3.1.4)}
Мы живём в некотором вероятностном пространстве $(\Omega, \mathcal{F}, P)$. Надо понимать, что у Круглова не случайный вектор является набором случайных величин, а случайная величина является одномерным случайным вектором. А случайный вектор - это измеримое отображение $\Omega \rightarrow R^d$. Измеримая функция же в свою очередь определяется точно так же, как мы привыкли, только вместо $\mathcal{B}(R)$ берётся $\mathcal{B}(R^d)$.
\begin{definition}
Произвольное семейство случайных векторов $X_t: \Omega \rightarrow R^d, t \in T$ называется \textit{случайным процессом}.
\end{definition}
\begin{definition}
\textit{Конечномерным распределением} случайного процесса называется мера $P_{t_1, \ldots, t_n}\{A\} = P\{(X_{t_1}, \ldots, X_{t_n}) \in A\}, A \in \mathcal{B}(R^{dn})$.
\end{definition}
Семейство конечномерных распределений является основной характеристикой случайного процесса. Дальше в билете формулируем две теоремы о них: простую и фундаментальную(ну ясен пень, это ж Колмогоров! Кто-нибудь видел не оч важную теорему Колмогорова?). Фундаментальность заключается в том, что она, по сути, гласит, что по заданным конечномерным распределениям можно построить случайный процесс.
\begin{theorem}
Конечномерные распределения удовлетворяют условиям согласованности:
$$P_{t_1, \ldots, t_n}\{\times_{k=1}^n A_k\} = P_{t_{\pi(1)}, \ldots, t_{\pi(n)}}\{\times_{k=1}^n A_{\pi(k)}\}$$.
$$P_{t_1, \ldots, t_{n+1}}\{\times_{k=1}^n A_k \times R^d\} = P_{t_1, \ldots, t_n}\{\times_{k=1}^n A_k\}$$, где, очевидно, $t_i \in T$ - любые, $A_i \in \mathcal{B}(R^d)$, $\pi$ - перестановка.
\end{theorem}
\begin{proof}[Идея доказательства]

\end{proof}
\begin{theorem}[Колмогоров]
Пусть семейство вероятностей $P_{t_1, \ldots, t_n}\{A\}, A \in \mathcal{B}(R^{dn})$ удовлетворяет условиям согласованности из предыдущей теоремы. Тогда существуют вероятность $P^T: \mathcal{B}((R^d)^T) \rightarrow [0, 1]$ и случайный процесс $X = {X_t, t \in T}$, определённый на вероятностном пространстве $((R^d)^T, \mathcal{B}((R^d)^T), P^T)$, такие, что $$P_{t_1, \ldots, t_n}^T\{A\} = P_{t_1, \ldots, t_n}\{A\}, \forall t_i \in T, A \in \mathcal{B}(R^{dn})$$.
\end{theorem}
\begin{proof}[Идея доказательства]
Оно обосраться какое здоровое, страницы 3, но по сути строится мера $\mu\{C_U(A)\} = P_{t_1, \ldots, t_n}\{A\}$ на алгебре цилиндрических множеств с конечномерными основаниями $A$, проверяется, что это действительно мера, и потом строится искомая $P^T$ как продолжение этой меры. Построив вероятность, процесс строится тупо полагая $X_t(\omega) = \omega_t, \forall \omega \in (R^d)^T, t \in T$.
\end{proof}

\section{Эквивалентные, неотличимые, одинаково распределенные, непрерывные случайные процессы (3.1.6 - 3.1.12)}
\begin{definition}
Случайные процессы $\{X_t, t \in T\}$ на $(\Omega, \mathcal{F}, P)$ и $\{X'_t, t \in T\}$ на $(\Omega', \mathcal{F}', P')$ называются \textit{одинаково распределёнными}, если для любых $t_1, \ldots, t_n, A$ выполнено $$P\{(X_{t_1}, \ldots, X_{t_n}) \in A\} = P'\{(X'_{t_1}, \ldots, X'_{t_n}) \in A\}$$.
\end{definition}
\begin{definition}
Пусть случайные процессы $X_t$, $X'_t$ определены на одном вероятностном пространстве и принимают значения в $R^d$. Если $\forall t \in T : P\{X_t \neq X'_t\} = 0$, то эти процессы называются \textit{эквивалентными}. Эквивалентные случайные процессы называются \textit{версиями} друг друга. 
\end{definition}
Если интерпретировать $T$ как время, то эквивалентность означает равенство почти наверное в любой фиксированный момент времени. Понятно, что эквивалентные процессы одинаково распределены.
\begin{definition}
Пусть случайные процессы $X_t$, $X'_t$ определены на одном вероятностном пространстве и принимают значения в $R^d$. Пусть есть некое $\Omega'$ такое, что $P\{\Omega'\} = 1$ и $\forall \omega \in \Omega'$ совпадают траектории $X_t(\omega)$ и $X'_t(\omega)$. Такие случайные процессы называются \textit{неотличимыми}.
\end{definition}
Неотличимые случайные процессы эквивалентны. Неотличимость -- самое сильное из возможных свойство двух процессов, далее эквивалентность и только потом одинаковая распределённость. Однако, если потребовать некоторые дополнительные условия на процессы и/или множество $T$, то можно показать, что и из эквивалентности следует неотличимость. Этому посвящены следующие две теоремы.
\begin{theorem}
Эквивалентные процессы со счётным множеством $T$ неотличимы.
\end{theorem}
\begin{proof}[Идея доказательства]
В качестве $\Omega'$ из определения неотличимых процессов возьмём $\cap_{t \in T}\{X_t = X'_t\} \in \mathcal{F}$.
\end{proof}
Для следующей теоремы понадобится ещё одно
\begin{definition}
Случайный процесс называется (\textit{почти})\textit{непрерывным} (\textit{непрерывным слева/справа}), если (почти) все его траектории непрерывны (непрерывны слева/справа).
\end{definition}
\begin{theorem}
Если эквивалентные процессы почти всюду непрерывны слева/справа, а множество $T$ выпукло, то они неотличимы.
\end{theorem}
\begin{proof}[Идея доказательства]
Берём $\Omega''$ как в предыдущей теореме, только на $T \cap Q$, где $Q$ - множество рациональных чисел. Для $\Omega''$ всё хорошо. Все остальные t приближаем последовательностью $\{t_n\}, t_i \in T \cap Q$. 
\end{proof}

\section{Стохастически непрерывные случайные процессы (3.1.13 - 3.1.14)}
Я очень сильно подозреваю, что $t^*$ и $t_*$ - это супремум и инфимум $T$ соответственно. Очень надеюсь, что это правда.
\begin{definition}
Случайный процесс $X = {X_t, t \in T}$ с выпуклым множеством $T$ называется \textit{стохастически непрерывным слева}, если $$\lim_{t \uparrow s}P\{\|X_t - X_s\| > \epsilon\} = 0, \forall \epsilon > 0, s > t_*$$.
\textit{Стохастическая непрерывность справа}: $$\lim_{t \downarrow s}P\{\|X_t - X_s\| > \epsilon\} = 0, \forall \epsilon > 0, s < t^*$$.
Случайный процесс называется \textit{стохастически непрерывным}, если он стохастически непрерывен слева и справа.
\end{definition}
\begin{definition}
Случайный процесс называется \textit{равномерно стохастически непрерывным}, если $$\lim_{h \rightarrow 0}\sup_{|t - s| < h}P\{\|X_t - X_s\| > \epsilon\} = 0, \forall \epsilon > 0$$
\end{definition}
\begin{theorem}
Стохастически непрерывный случайный процесс равномерно стохастически непрерывен.
\end{theorem}
\begin{proof}[Идея доказательства]

\end{proof}

\section{Теорема существования сепарабельных случайных процессов (3.2.1 - 3.2.5)}
Рассматривается задача: вычислить вероятность того, что траектории случайного процесса лежат в данном множестве. Фишка в том, что эта задача может оказаться некорректной для некоторого процесса $X$ (например для $X_t(\omega) = 1$ при $\omega \in A$, иначе $0$, для неборелевского $A \subset [0, 1]$). С другой стороны, можно взять процесс $Y$, эквивалентный $X$ (в примере выше просто $Y_t = 0$), для которого задача будет звучать корректно. Это утверждение верно для любого $X$ и является теоремой, которую мы сформулируем дальше (доказана Дубом. Даже Дуб что-то может доказать, а ты нет...). Процесс $Y$ называют \textit{сепарабельной версией} $X$. Определим теперь всё более формально.

Кроме уже надоевшего процесса $X$, определённого на вероятностном пространстве $(\Omega, \mathcal{F}, P)$ со значениями в $R^d$, введём ещё обозначение $\overline{X(U, \omega)}$. Это замыкание множества $X(U, \omega) = \{X_t(\omega) \mid t \in U\}$. Кроме того, назовём \textit{относительным интервалом} пересечение $(a, b) \cap T$ вещественного интервала $(a, b)$ и множества $T$.

\begin{definition}
Случайный процесс называется \textit{сепарабельным}, если существует конечное или счётное $S \subset T$ и событие $N \in \mathcal{F}, P\{N\} = 0$ такие, что $$X_u(\omega) \in \cap_{J: u \in J} \overline{X(J \cap S, \omega)}$$ для любых $u \in T$ и $\omega \notin N$, а $J$ - относительный интервал, содержащий $u$. Пересекаем по всем таким $J$. $S$ называется \textit{сепарантой}, а $N$ - \textit{исключительным событием} случайного процесса.
\end{definition}
Смысл условия в определении заключается в том, что почти каждая траектория случайного процесса определяется своими значениями на счётном множестве $S$. Понятно, что процесс может быть несепарабельным только в случае несчётного $T$, иначе можно просто взять $S = T$. Верно также утверждение, что почти всюду непрерывный слева/справа процесс с выпуклым $T$ является сепарабельным.
Следующая теорема является критерием сепарабельности случайного процесса.
\begin{theorem}
Случайный процесс сепарабельный $\Longleftrightarrow$ существуют счётное множество $S$, событие нулевой вероятности $N$ такие, что для любого замкнутого множества $K \subset \overline{R}^d$ и относительного интервала $J$ $$\cap_{t \in J \cap S}\{X_t \in K\} \subseteq \cap_{t \in J}\{X_t \in K\} \cup N$$ или (эквивалентное условие) $$\cap_{t \in J \cap S}\{X_t \in K\} \subseteq \{X_u \in K\} \cup N, \forall u \in J$$.
\end{theorem}
\begin{proof}[Идея доказательства]

\end{proof}
\begin{theorem}
Найдутся счётное множество $S \subset T$ и события $N^u \in \mathcal{F}, u \in T$ нулевой вероятности такие, что $$X_u(\omega) \in \cap_{J: u \in J} \overline{X(J \cap S, \omega)}$$ для любых $u \in T$ и $\omega \notin N^u$.
\end{theorem}
Теперь самое время для, собственно, основной теоремы. Предварительно только определим \textit{сепарабельную версию} процесса $X$, как некий эквивалентный процесс $Y$, обладающий свойством сепарабельности. 
\begin{theorem}
Любой случайный процесс $X$ имеет сепарабельную версию $Y$, $Y_t:\Omega \rightarrow \overline{R}^d$.
\end{theorem}
\begin{proof}[Идея доказательства]

\end{proof}
Последнее утверждение касается сепарант сепарабельных стохастически непрерывных процессов с выпуклым множеством $T$.
\begin{theorem}
Если процесс сепарабельный и стохастически непрерывный с выпуклым множеством $T$, то в качестве сепаранты можно взять любое всюду плотное множество счётное множество $S \subset T$.
\end{theorem}
\begin{proof}[Идея доказательства]

\end{proof}

\section{Свойства вещественных сепарабельных процессов (3.2.6)}
Обозначим $J(T)$ - класс всех относительных интервалов множества $T \subseteq R$, $\mathcal{E}$ - класс множеств вида $[-\infty, a], [a, b], [a, \infty], a \le b, a,b \in R$. В формулировке следующей теоремы выбираются любые множества: $J \in J(T), K \in \mathcal{E}, u \in J, \omega \notin N$, где $N$ - исключительное событие. 
\begin{theorem}
Пусть всё так же есть вещественный случайный процесс $X$, который ещё и сепарабелен с сепарантой $S$ и исключительным событием $N$. Тогда следующие утверждения эквивалентны:
\begin{enumerate}
    \item $\cap_{t \in J \cap S}\{X_t \in K\} \subseteq \cap_{t \in J}\{X_t \in K\} \cup N$
    \item $\inf_{t \in J}X_t(\omega) = \inf_{t \in J \cap S}X_t(\omega)$; \newline $\sup_{t \in J}X_t(\omega) = \sup_{t \in J \cap S}X_t(\omega)$
    \item $\inf_{t \in J \cap S}X_t(\omega) \le X_u(\omega) \le \sup_{t \in J \cap S}X_t(\omega)$
    \item $\lim_{J \ni t \rightarrow u} \inf X_t(\omega) = \lim_{J \cap S \ni t \rightarrow u} \inf X_t(\omega)$; \newline $\lim_{J \ni t \rightarrow u} \sup X_t(\omega) = \lim_{J \cap S \ni t \rightarrow u} \sup X_t(\omega)$
    \item $\lim_{J \cap S \ni t \rightarrow u} \inf X_t(\omega) \le X_u(\omega) \le \lim_{J \cap S \ni t \rightarrow u} \sup X_t(\omega)$
\end{enumerate}
\end{theorem}
Эти пять условий на самом деле довольно просто запоминаются мнемонически, достаточно помнить, что 1 - это просто из теоремы предыдущего билета, 2 - два одинаковых утверждения для точных нижних и верхних граней, меняется (сужается) лишь множество, в котором мы смотрим $t$, третье - тупо неравенство для инфимума и супремума из второго, а 4 и 5 - по сути 2 и 3, только с добавленными пределами при $t \rightarrow u$.
\begin{proof}[Идея доказательства]
Доказываем цепочку следствий: $1 \Longrightarrow 2 \Longrightarrow 3 \Longrightarrow 4 \Longrightarrow 5 \Longrightarrow 1$.
\end{proof}

\section{Достаточные условия непрерывности случайных процессов (3.3.1 - 3.3.4)}
\begin{theorem}
Сепарабельный случайный процесс с выпуклым множеством $T$ почти всюду непрерывен, если $$\lim_{h \downarrow 0}P\{\sup_{|s - t| < h} \|X_s - X_t\| > \epsilon\} = 0, \forall \epsilon > 0$$.
\end{theorem}
\begin{proof}[Идея доказательства]
То, что должно быть больше эпсилона, -- убывающая последовательность $Z_h$. Тогда есть её предел. В силу условия теоремы, этот предел $Z$ равен нулю почти всюду. Тогда траектории непрерывны в любом $\omega \in Z$.
\end{proof}
\begin{theorem}
Сепарабельный случайный процесс с $T = [a, b]$ почти всюду непрерывен $\Longleftrightarrow$ выполнено условие из предыдущей теоремы.
\end{theorem}
\begin{proof}[Идея доказательства]
По предыдущей теореме следствие в одну сторону верно. В другую пользуемся теоремой Кантора, по которой из непрерывности почти всюду следует равномерная непрерывность почти всюду, а из сходимости почти всюду вытекает сходимость по вероятности.
\end{proof}
Разница между первой и следующей теоремой в том, что в первой мы берём любые $s, t \in T$ внутри вероятности, а в этой смотрим супремум по $s$ вероятностей, где $s$ уже фиксированное.
\begin{theorem}
Сепарабельный случайный процесс с выпуклым множеством $T$ почти всюду непрерывен, если $$\lim_{h \downarrow 0}\frac{1}{h}\sup_{s \in T}P\{\sup_{|s - t| < h} \|X_s - X_t\| > \epsilon\} = 0, \forall \epsilon > 0$$
\end{theorem}
\begin{proof}[Идея доказательства]

\end{proof}

Последняя теорема - какой-то пи**ец, запомнить можно, доказывать не советую. Выглядит полезно и классно, и оценка скорости равномерной сходимости, и все дела, но это всё вплоть до доказательства.
\begin{theorem}
Пусть случайный процесс с выпуклым множеством $T$ удовлетворяет условию: $$P\{\|X_s - X_t\| \ge p(|t - s|)\} \le q(|t - s|)$$ для любых $s, t \in T : |t - s| < \delta, \delta > 0$ и для некоторых функций $p,q: [0, \delta] \rightarrow R_+$ таких, что $$\int_0^\delta \frac{p(u)}{u} du < \infty, \int_0^\delta \frac{q(u)}{u^2} du < \infty.$$ Тогда существует непрерывная версия $Y$ процесса $X$. Более того, для любого конечного отрезка $[a, b] \subseteq T$ существует функция $H: \Omega \rightarrow \{1, 2, \ldots\}$ такая, что $\forall \omega \in \Omega, h \in (0, 2^{-H(\omega)} \wedge \delta)$ выполнено неравенство $$\sup_{s, t \in [a, b]: |t - s| < \frac{h}{2}} \|Y_s(\omega) - Y_t(\omega)\| \le \frac{2}{\ln 2} \int_0^h \frac{p(u)}{u} du$$
\end{theorem}
\begin{proof}[Идея доказательства]
Эта жопа разбивается на 5 пунктов, каждый из которых всё больше обобщает теорему.
\end{proof}

\section{Теорема Колмогорова о непрерывных случайных процессах (3.3.5)}
Какая-то странная теорема Колмогорова... Во-первых потому, что Круглов не упомянул в учебнике, что это именно Колмогоров, во-вторых почему-то она не очень фундаментально звучит, в-третьих, всего-то страница доказательства, а я с трудом верю, что Колмогоров называл своим именем то, что Тыртышникову очевидно. Ну да ладно, собственно, сама
\begin{theorem}[Колмогоров]
Если случайный процесс с выпуклым множеством $T$ удовлетворяет условию $$\mathsf{E} \|X_t - X_s\|^\alpha \le c|s - t|^{1 + \beta}$$ для любых $s, t$ и каких-то положительных $a, b, c$, то он имеет непрерывную версию $Y$ со свойством $$\lim_{h \downarrow 0}\frac{1}{h^\gamma}\sup_{s, t \in [a, b]: |t - s| < h} \|Y_t - Y_s\| = 0$$ для любого $\gamma \in (0, \frac{\beta}{\alpha})$ и любого сегмента $[a, b] \in T$.
\end{theorem}
\begin{proof}[Идея доказательства]

\end{proof}

\section{Функции без разрывов второго рода (3.4.1 - 3.4.5)}
\begin{definition}
Функция $f: T \rightarrow R^d$ \textit{не имеет разрывов второго рода}, если есть $f(t-) \forall t > t_*$ и $f(t+) \forall t < t^*$.
\end{definition}
Доопределим $f(t_*-) = f(t_*), f(t^*+)=f(t^*)$. Разность $\Delta f(t) = f(t+) - f(t-)$ назовём \textit{скачком} функции $f$ в точке $t \in T$, а $\|\Delta f(t)\|$ - \textit{величиной скачка}.
\begin{theorem}
Пусть дана функция без разрывов второго рода. Тогда $\forall c > 0, a, b \in T, a < b$ множество $E_{c, a, b} = \{t \in [a, b] \mid \|\Delta f(t)\| \ge c\}$ конечно, а множество $E = \{t \in [a, b] \mid \|\Delta f(t)\| \ge 0\}$ конечно или счётно.
\end{theorem}
\begin{proof}[Идея доказательства]

\end{proof}
\begin{theorem}
Если нет разрывов второго рода, то $\sup_{a \le t \le b} \|f(t)\| \le \infty, \forall a, b \in T$. Если, ко всему прочему, ещё и $\|\Delta f(t)\| \le c$ для некоторого $c > 0$ и всех $t$, то $\forall \epsilon > 0, a, b \exists \delta(\epsilon, a, b) > 0$ такое, что $\|f(t) - f(s)\| < c + \epsilon$ при любых $s, t \in [a, b], |t - s| < \delta$.
\end{theorem}
По сути утверждается, что точная верхняя грань конечна, а если скачки ограничены какой-то константой, то на любом сегменте для наперёд заданного эпсилона можно выбрать дельту так, чтобы функция за время, не большее, чем дельта, (всё ещё интерпретируем $T$ как время) изменилась меньше, чем на ограничивающую константу плюс эпсилон.
\begin{proof}[Идея доказательства]

\end{proof}
\begin{definition}
Функция называется \textit{регулярной справа(слева)}, если она непрерывна справа(слева) в каждой точке $t \in T, t < t^* (t > t_*)$ и имеет предел слева(справа) в каждой точке $t \in T, t > t_* (t < t^*)$. 
\end{definition}
\begin{theorem}
Пусть функция определена на всюду плотном подмножестве $S$ множества $T$. Предположим, что существуют $\lim_{S \ni s \uparrow t} f(s) = g(t-) \in R^d, \forall t \in T, t > t_*$ и $\lim_{S \ni s \downarrow t} f(s) = g(t+) \in R^d, \forall t \in T, t < t^*$. Тогда функция $g(t-), t \in T, t > t_*$ регулярна слева, функция $g(t+), t \in T, t < t^*$ регулярна справа и $\sup_{t \in [a, b] \cap S} \|f(t)\| < \infty, \forall a, b \in T, a < b$.
\end{theorem}
\begin{proof}[Идея доказательства]

\end{proof}

\section{Случайные процессы без разрывов второго рода: знать формулировки теорем (3.4.7
- 3.4.8)}
Спасибо Круглову за отсутствие необходимости знать доказательства, в противном случае можно было бы сразу идти на пересдачу. 
\begin{theorem}
Пусть случайный процесс с выпуклым множеством $T, t_* \in T$ удовлетворяет условию $$P\{\|X_{t_1} - X_{t_2}\|\|X_{t_2} - X_{t_3}\| \ge p(t_3 - t_1)\} \le q(t_3 - t_1)$$ для любых $t_i \in T, t_1 < t_2 < t_3, t_3 - t_1 < \delta, \delta > 0$ и для некоторых неубывающих функций $p,q: [0, \delta] \rightarrow R_+$ таких, что $$\int_0^\delta \frac{p(u)}{u} du < \infty, \int_0^\delta \frac{q(u)}{u^2} du < \infty.$$ Тогда процесс $X$ имеет версию $Y$ без разрывов второго рода.
\end{theorem}
Теорема выше является обобщённым случаем второй теоремы.
\begin{theorem}[Ченцов]
Пусть случайный процесс с выпуклым множеством $T, t_* \in T$ удовлетворяет условию $$\mathsf{E}(\|X_{t_1} - X_{t_2}\|\|X_{t_2} - X_{t_3}\|)^\alpha \le c|t_3 - t_1|^{1 + \beta}$$ для некоторых $\alpha, \beta > 0, t_i \in T, t_1 < t_2 < t_3$. Тогда процесс $X$ имеет версию $Y$ без разрывов второго рода. 
\end{theorem}
В некотором роде эти две теоремы являются аналогами теорем 8.4 и 9.1(Колмогорова), только там непрерывные, а здесь без разрывов второго рода.

\

\section{Фильтрации и их свойства, естественные фильтрации случайных процессов (3.5.1
- 3.5.6)}
\begin{definition}
Семейство $\mathrm{F}_T = \{\mathcal{F}_t \mid \mathcal{F}_t \subseteq \mathcal{F}, t \in T\}$ сигма-алгебр называется \textit{фильтрацией}, если $\mathcal{F}_s \subseteq \mathcal{F}_t$ для любых $s < t, s, t \in T$.
\end{definition}
\begin{definition}
Фильтрация называется \textit{расширенной}, если любое множество $A \in \mathcal{F}$ нулевой вероятности принадлежит всем сигма-алгебрам $\mathcal{F}_t$.
\end{definition}

Введём обозначения $\mathcal{F}_{t+} = \cap_{s > t} \mathcal{F}_s, \mathcal{F}_{t-} = \sigma(\mathcal{F}_s, s < t)$.
\begin{definition}
Фильтрация называется \textit{непрерывной справа(слева)}, если $\mathcal{F}_t = \mathcal{F}_{t+} (\mathcal{F}_{t-} = \mathcal{F}_t), \forall t \in T$.
\end{definition}
Фильтрация со счётным $T$ непрерывна слева $\Longleftrightarrow$ все сигма-алгебры равны между собой.

Пусть дана фильтрация $\mathrm{F}_T$ с выпуклым параметрическим множеством $T$. Введём следующие классы: $\mathcal{N} = \{A \mid A \in \mathcal{F}, P(A) = 0\}, \mathcal{G}_t = \sigma(\mathcal{F}_t, \mathcal{N}), \mathcal{G}_{t+} = \cap_{s > t} \mathcal{G}_s, \forall t < t^*, \mathcal{G}_{t^*+} = \mathcal{G}_{t^*}$.
\begin{theorem}
В обозначениях выше, для определённой выше фильтрации $\mathrm{F}_T$ справедливо, что фильтрация $\mathcal{G}_{T+}$ непрерывна справа, расширена и обладает минимальным свойством в том смысле, что $\mathcal{G}_{t+} \subseteq \mathcal{H}_t, \forall t \in T$ для любой расширенной, непрерывной справа фильтрации $\{\mathcal{H}_t \mid \mathcal{F}_t \subseteq \mathcal{H}_t, \forall t \in T\}$.
\end{theorem}
\begin{proof}[Идея доказательства]

\end{proof}
\begin{definition}
Случайный процесс называется \textit{согласованным} с фильтрацией, если для любого $t$ случайный вектор $X_t$ измерим относительно $\mathcal{F}_t$.
\end{definition}
\begin{definition}
Фильтрация $\mathcal{F}_T^{(X)}, \mathcal{F}_t^{(X)} = \sigma(X_s, s \le t)$ называется \textit{естественной фильтрацией} случайного процесса $X$.
\end{definition}
\begin{definition}
Фильтрация $\mathcal{G}_T^{(X)}, \mathcal{G}_t^{(X)} = \sigma(\mathcal{F}_t^{(X)}, \mathcal{N})$ называется \textit{расширенной естественной фильтрацией} случайного процесса $X$.
\end{definition}
Случайный процесс согласован с каждой из своих естественных фильтраций. Если вспомнить, что мы привыкли интерпретировать $T$ как время, то $\mathcal{F}_t^{(X)}$ содержит информацию о поведении процесса вплоть до времени $t \in T$.
\begin{theorem}
Если случайный процесс с выпуклым множеством $T$ непрерывен слева, то его естественные фильтрации $\mathcal{F}_T^{(X)}, \mathcal{G}_T^{(X)}$ тоже непрерывны слева.
\end{theorem}
\begin{proof}[Идея доказательства]

\end{proof}
\begin{theorem}
Если случайный процесс с выпуклым множеством $T$ стохастически непрерывен слева, то его расширенная естественная фильтрация $\mathcal{G}_T^{(X)}$ тоже непрерывна слева.
\end{theorem}
\begin{proof}[Идея доказательства]

\end{proof}

\section{Марковские моменты (3.6.1 - 3.6.7)}
\begin{definition}
Функция $\tau: \Omega \rightarrow T \cup \{\infty\}$ называется \textit{$\mathrm{F}_T$-марковским моментом} или \textit{марковским моментом относительно фильтрации $\mathrm{F}_T$}, если $\{\tau \le t\} \in \mathcal{F}_t, \forall t \in T$.
\end{definition}
\begin{example}
Функция, тождественная равная $u \in T$, является марковским моментом, потому что множество $\{\tau \le t\}$ либо $\emptyset$, либо $\Omega$, а оба они лежат в любой сигма-алгебре.
\end{example}
\begin{theorem}
Если $\tau$ - марковский момент, то $\{\tau < t\} \in \mathcal{F}_t, \forall t \in T$. 
\end{theorem}
\begin{proof}[Идея доказательства]

\end{proof}
\begin{theorem}
Пусть множество $T$ - счётное. Тогда функция $\tau$ является марковским моментом $\Longleftrightarrow \{\tau = t\} \in \mathcal{F}, \forall t \in T$.
\end{theorem}
\begin{proof}[Идея доказательства]

\end{proof}
\begin{theorem}
Функция $\tau$ является марковским моментом относительно непрерывной справа фильтрации с выпуклым множеством $T$ $\Longleftrightarrow \{\tau < t\} \in \mathcal{F}_t, \forall t \in T$ и $\{\tau = \infty\} \in \mathcal{F}_{t^*}$, если $t^* \in T$.
\end{theorem}
\begin{proof}[Идея доказательства]

\end{proof}
\begin{theorem}
Пусть $\tau, \sigma: \Omega \rightarrow T \cup \{\infty\}$. Если $\tau$ - марковский момент относительно расширенной фильтрации $\mathrm{F}_t$ и почти всюду $\tau = \sigma$, то $\sigma$ - марковский момент относительно $\mathrm{F}_t$.
\end{theorem}
\begin{proof}[Идея доказательства]

\end{proof}
\begin{theorem}
Если $\tau$ - марковский момент, то функция $\tau \wedge t$ измерима относительно $\mathcal{F}_t$ для любого $t \in T$.
\end{theorem}
\begin{proof}[Идея доказательства]

\end{proof}
\begin{theorem}
Если $\tau, \sigma$ - марковские моменты, то $\tau \wedge \sigma, \tau \vee \sigma$ - тоже марковские моменты.
\end{theorem}
\begin{proof}[Идея доказательства]

\end{proof}

\section{Сигма-алгебры, связанные с марковскими моментами (3.6.8 - 3.6.9)}
\begin{definition}
С каждым $\mathrm{F}_T$-марковским моментом $\tau$ связаны две $\sigma$-алгебры $\mathcal{F}_\tau$ и $\mathcal{F}_{\tau-}$. \newline
Сигма-алгебра $\mathcal{F}_\tau$ состоит из множеств $A \in \sigma(\mathcal{F}_t, t \in T)$, для которых $A \cap \{\tau \le t\} \in \mathcal{F}_t$. Она называется \textit{сигма-алгеброй событий, предшествующих марковскому моменту $\tau$}. \newline
Сигма-алгебра $\mathcal{F}_{\tau-}$ порождается множествами $A \cap \{t < \tau\}, A \in \mathcal{F}_t$. Если дополнительно $t_* \in T$, то к классу порождающих множеств добавляются ещё все множества $A \in \mathcal{F}_{t_*}$.
\end{definition}
\begin{theorem}
Пусть $\tau, \sigma$ - марковские моменты. Тогда верны следующие утверждения:
\begin{enumerate}
    \item $\mathcal{F}_{\tau-} \subseteq \mathcal{F}_\tau$.
    \item $\mathcal{F}_{\tau} \subseteq \mathcal{F}_\sigma, \mathcal{F}_{\tau-} \subseteq \mathcal{F}_{\sigma-}$ при $\tau \le \sigma$.
    \item $\mathcal{F}_{\tau \wedge \sigma} = \mathcal{F}_\tau \cap \mathcal{F}_\sigma$.
    \item $\mathcal{F}_{\tau} \subseteq \mathcal{F}_{\sigma-}$ при $\tau < \sigma$.
    \item $\{\tau \le \sigma\}, \{\tau = \sigma\} \in \mathcal{F}_{\tau \wedge \sigma}$ \newline $\{\tau < \sigma\} \in \mathcal{F}_\tau \cap \mathcal{F}_{\sigma-}$.
    \item Если $A \in \mathcal{F}_\tau$, то $A \cap \{\tau \le \sigma\}, A \cap \{\tau = \sigma\} \in \mathcal{F}_\sigma$.
    \item Если $A \in \mathcal{F}_\tau$, то $A \cap \{\tau < \sigma\} \in \mathcal{F}_\tau \cap \mathcal{F}_{\sigma-}$.
\end{enumerate}
\end{theorem}
\begin{proof}[Идея доказательства]

\end{proof}

\section{Измеримость марковских моментов и другие из свойства (3.6.10 - 3.6.15)}
\begin{theorem}
Любой марковский момент $\tau$ измерим относительно $\sigma$-алгебры $\mathcal{F}_{\tau-}$.
\end{theorem}
\begin{proof}[Идея доказательства]

\end{proof}
\begin{theorem}
Пусть дан марковский момент $\tau$ и $\mathcal{F}_\tau$-измеримая функция $\sigma: \Omega \rightarrow T \cup \{\infty\}$. Тогда:
\begin{enumerate}
    \item Если $\tau \le \sigma$, то $\sigma$ - марковский момент.
    \item $\forall A \in \mathcal{F}_\tau$ функция $\tau_A = \tau\mathbf{1}_A + \infty\mathbf{1}_{A^c}$ является марковским моментом.
\end{enumerate}
\end{theorem}
\begin{proof}[Идея доказательства]

\end{proof}
\begin{theorem}
Для любого марковского момента $\tau$ относительно фильтрации $\mathrm{F}_T$ с выпуклым множеством $T$ существуют такие марковские моменты $\tau_n$, что каждый из них принимает конечное число значений и $\tau_n \downarrow \tau$.
\end{theorem}
\begin{proof}[Идея доказательства]

\end{proof}
\begin{theorem}
Пусть дана последовательность марковских моментов $\{\tau_n\}$. Если она убывает и $\forall \omega \in \Omega$ найдётся номер $m$, начиная с которого все $\tau_n(\omega) = \tau_m(\omega)$, то функция $\tau = \lim_{n \rightarrow \infty}\tau_n$ является марковским моментом.
\end{theorem}
\begin{proof}[Идея доказательства]

\end{proof}
\begin{theorem}
Пусть даны марковские моменты $\tau_n$. Если $\forall t_n \in T \sup_{n \rightarrow \infty}t_n \in T \cup \{\infty\}$, то $\tau = \sup_{n \rightarrow \infty}\tau_n$ - марковский момент и $\cup_{n = 1}^\infty \mathcal{F}_{\tau_n} \subseteq \mathcal{F}_\tau$.
\end{theorem}
\begin{proof}[Идея доказательства]

\end{proof}
\begin{theorem}
Пусть даны марковские моменты $\tau_n$ относительно фильтрации $\mathrm{F}_T$. Если фильтрация непрерывна справа, а $T = {1, 2, \ldots}$ или $T = [a, b]$, или $T = [a, \infty)$, то функции $$\lim\sup \tau_n, \lim\inf \tau_n, \sup \tau_n, \inf \tau_n, n \rightarrow \infty$$ являются марковскими моментами и $\mathcal{F}_{\inf \tau_n} = \cap_{n = 1}^\infty \mathcal{F}_{\tau_n}$.
\end{theorem}
\begin{proof}[Идея доказательства]

\end{proof}

\section{Предсказуемые марковские моменты (3.7.1 - 3.7.7)}
Теперь у нас $T = [0, \infty)$ и фильтрация $\mathrm{F} = \{\mathcal{F}_t \mid \mathcal{F}_t \subseteq \mathcal{F}, t \ge 0\}$. Соответственно, все марковские моменты берутся относительно этой фильтрации, теперь это функции вида $\tau: \Omega \rightarrow \overline{R}_+$.
\begin{definition}
Марковский момент называется \textit{предсказуемым}, если $\exists \tau_n\uparrow, \tau_n < \tau$ на множестве $\{\tau > 0\}$ такая, что $\lim_{n \rightarrow \infty} \tau_n = \tau$. Сама последовательность называется \textit{предвещающей}.
\end{definition}
Дальше идут три примера:
\begin{enumerate}
    \item $\forall \alpha \ge 0, A \in \mathcal{F}_\alpha: \tau = \alpha\mathbf{1}_A + \infty\mathbf{1}_{A^c}$ - предсказуемый марковский момент. Марковский момент - потому что теорема в предыдущем билете, предсказуемый - потому что если $\alpha > 0$, то есть $\alpha_n \uparrow$ такая, что $\alpha_n \rightarrow \alpha$, ну и предвещающей последовательностью берём $\{\tau_n \wedge n\}$, где $\tau_n = \alpha_n\mathbf{1}_A + \infty\mathbf{1}_{A^c}$. А если $\alpha = 0$, то возьмём $\{\tau_A \wedge n\}$.
    \item Сумма $\tau + \sigma$, где $\tau$ - марковский момент, а $\sigma$ - предсказуемый марковский момент, является предсказуемым марковским моментом. В качестве предвещающей последовательности берём $\{\tau + \sigma_n\}$, где $\sigma_n$ - предвещающая последовательность для $\sigma$. Каждая из таких сумм - марковский момент, потому что была такая задача в предыдущем параграфе, причём без решения, так что, видимо, просто потому что.
    \item $X$ - случайный процесс: $\mathrm{F}$-согласованный, непрерывный, вещественный (нашедший своё место в жизни, самореализовавшийся, семья, дети, все дела. Кто-нибудь помнит, что эти слова вообще значат?). Тогда $\forall c \in R: \tau = \inf\{t \ge 0 \mid X_t \ge c\}$ - предсказуемый марковский момент. Потому что, бл*ть, гладиолус, дальше на страницу чё-то расписано, но сводится к тому, что предвещающая последовательность - $\{\tau_n \wedge n\}$, где $\tau_n = \inf\{t \ge 0 \mid X_t \ge c - \frac{1}{n}\}$. 
\end{enumerate}
Настало время теорем. Особенно неожиданным результатом кажется первое утверждение следующей теоремы.
\begin{theorem}
Предсказуемый марковский момент является марковским моментом; $\mathcal{F}_{\tau-} = \sigma(\cup_{n = 1}^\infty \mathcal{F}_{\tau_n})$ для любой предвещающей последовательности $\tau_n$.
\end{theorem}
\begin{proof}[Идея доказательства]

\end{proof}
\begin{theorem}
Пусть $\tau, \sigma$ - предсказуемые марковские моменты. Тогда:
\begin{enumerate}
    \item $\tau \wedge \sigma, \tau \vee \sigma$ - предсказуемые марковские моменты.
    \item $\{\tau \le \sigma\}, \{\tau < \sigma\}, \{\tau \ge \sigma\}, \{\tau > \sigma\}, \{\tau = \sigma\} \in \mathcal{F}_{\tau-}\cap\mathcal{F}_{\sigma-}$.
    \item $A \cap \{\tau \le \sigma\}, A \cap \{\tau < \sigma\}, A \cap \{\tau = \sigma\} \in \mathcal{F}_{\tau-}\cap\mathcal{F}_{\sigma-}$ для любого $A \in \mathcal{F}_{\tau-}$
\end{enumerate}
\end{theorem}
\begin{proof}[Идея доказательства]

\end{proof}
\begin{theorem}
Пусть даны предсказуемые марковские моменты $\tau_n$.
\begin{enumerate}
    \item Если $\tau_n \uparrow$, то $\tau = \lim_{n \rightarrow \infty}\tau_n$ - предсказуемый марковский момент.
    \item Если $\tau_n \downarrow$ и для любого $\omega \in \Omega$ найдётся номер $m$, начиная с которого $\tau_n(\omega) = \tau_m(\omega)$, то $\tau = \lim_{n \rightarrow \infty}\tau_n$ - предсказуемый марковский момент.
\end{enumerate}
\end{theorem}
\begin{proof}[Идея доказательства]

\end{proof}
\begin{theorem}
Пусть $\tau$ - предсказуемый марковский момент. Тогда $\forall A \in \mathcal{F}_{\tau-}$ функция $\tau_A = \tau\mathbf{1}_A + \infty\mathbf{1}_{A^c}$ - предсказуемый марковский момент.
\end{theorem}
\begin{proof}[Идея доказательства]

\end{proof}
\end{document}
