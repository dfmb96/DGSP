\documentclass[16pt]{article}
\usepackage[utf8]{inputenc}
\usepackage[russian]{babel}
\usepackage[urlcolor=blue]{hyperref}
\usepackage{amsthm}

\newtheorem{theorem}{Теорема}[section]
\newtheorem{lemma}[theorem]{Лемма}
\theoremstyle{definition}
\newtheorem{definition}[theorem]{Определение}
\newtheorem{example}{Пример}[theorem]

\begin{document}

\href{https://yadi.sk/d/d-ti\_TZi3Mh3Ri/7\%20sem/\%D0\%94\%D0\%93\%D0\%A1\%D0\%9F}{Здесь лежит учебник Круглова, издание 2016 года.}

\newpage

\section{Теорема Серпинского (1.2.7)}
Обозначение класса всех подмножеств: $2^\Omega$.
\begin{definition}
Класс $\mathcal{E} \in 2^\Omega$ называется 
\begin{itemize}
    \item \textit{$\pi$-классом}, если он замкнут относительно пересечения своих элементов ($\forall A, B \in \mathcal{E} : A \cap B \in \mathcal{E}$)
    \item \textit{алгеброй}, если $\Omega, A \cup B, A^c \in \mathcal{E}$
    \item \textit{$\sigma$-алгеброй}, если он является алгеброй и содержит счётное объединение любых своих подмножеств. $\sigma$-алгеброй, \textit{порождённой} $L$, называется минимальная по включению $\sigma$-алгебра, содержащая $L$. Обозначается $\sigma(L)$. 
    \item \textit{монотонным}, если $\forall A_n \uparrow$ : $\cup_{i=1}^\infty A_i \in \mathcal{E}$, $\forall A_n \downarrow$ : $\cap_{i=1}^\infty A_i \in \mathcal{E}$
    \item \textit{$\lambda$-классом}, если $\Omega, A \backslash B, \cup_{i=1}^\infty A_i \in \mathcal{E}$ при $B \subseteq A, A_n \uparrow$.
\end{itemize}
\end{definition}
Если класс является $\pi$- и $\lambda$-классом, то он $\sigma$-алгебра. Кроме того, все эти специальные классы замкнуты относительно пересечения, то есть любое пересечение, например, алгебр является алгеброй. \newline
Если $L$ - один из специальных классов, определённых выше, то $L \cap B$ - класс того же типа, $\forall B \in 2^\Omega$. Кроме того, если $L$ - образ некоторого отображения, то его прообраз также является классом того же типа. Если же есть $L$ - прообраз некоторого отображения в $\Omega'$, то класс множеств из $\Omega'$, прообразы которых лежат в $L$, является классом того же типа.

Вообще, следующая теорема содержит два утверждения. Круглов пишет, что первое из них - теорема Серпинского, а второе - теорема о монотонном классе. Но раз они сформулированы вместе, думаю, что теормин этого билета содержит оба утверждения.
\begin{theorem}[Серпинский]
Справедливы следующие утверждения:
\begin{enumerate}
    \item Если $\pi$-класс $\mathcal{E}$ содержится в $\lambda$-классе $\mathcal{D}$, то $\sigma(\mathcal{E}) \subseteq \mathcal{D}$.
    \item Если алгебра $\mathcal{A}$ содержится в монотонном классе $\mathcal{M}$, то $\sigma(\mathcal{A}) \subseteq \mathcal{M}$. 
\end{enumerate}
\end{theorem}
\begin{proof}[Идея доказательства]
\begin{enumerate}
    \item Пусть $\lambda(\mathcal{E})$ - минимальный $\lambda$-класс, содержащий $\mathcal{E}$. Для произвольного $A \in \mathcal{E}$ вводится класс $\mathcal{L}(A)$ множеств $B \in \lambda(\mathcal{E})$ таких, что $A \cap B \in \lambda(\mathcal{E})$. Доказывается, что $\mathcal{L}(A)$, является $\lambda$-классом и проверяется, что $\mathcal{E} \subseteq \mathcal{L}(A)$. Поскольку $\mathcal{L}(A) \subseteq \lambda(\mathcal{E})$, получаем, что $\mathcal{L}(A) = \lambda(\mathcal{E})$. \newline Теперь берём $B \in \lambda(\mathcal{E})$ и класс $\mathcal{M}(B)$ множеств $D$ таких, что $B \cap D \in \lambda(\mathcal{E})$. Это $\lambda$-класс и содержит $\pi$-класс $\mathcal{E}$. Откуда $\mathcal{M}(B) = \lambda(\mathcal{E})$. Тем самым доказано, что $\lambda(\mathcal{E})$ - $\pi$-класс. Тогда по утверждению до теоремы, $\lambda(\mathcal{E})$ - $\sigma$-алгебра. А дальше утверждается, что осталось заметить, что $\lambda(\mathcal{E}) \subseteq \lambda(\mathcal{D}) \subseteq \mathcal{D}$. 
    \item Похожие действия. Пусть $m(\mathcal{A})$ - минимальный монотонный класс, содержащий $\mathcal{A}$, далее, зафиксировав $A \in \mathcal{A}$, вводим класс $\mathcal{L}(A)$ множеств $B \in m(\mathcal{A})$ таких, что $B \backslash A \in m(\mathcal{A})$. Убеждаемся, что $\mathcal{L}(A)$ является монотонным классом, содержащим алгебру $\mathcal{A}$. Поэтому $m(\mathcal{A}) \subseteq \mathcal{L}(A) \subseteq \mathcal{M}$.\newline Теперь берём $B \in m(\mathcal{A})$, вводим класс $\mathcal{M}(B)$ множеств $D \in m(\mathcal{A})$ таких, что $D \backslash B \in m(\mathcal{A})$. Убеждаемся, что $\mathcal{M}(B)$ - тоже монотонный класс, содержащий алгебру $\mathcal{A}$. Тогда $m(\mathcal{A}) \subseteq \mathcal{M}(B)$. Поэтому $m(\mathcal{A})$ - $\lambda$-класс, содержащий $\pi$-класс $\mathcal{A}$. Тогда по первому утверждению сигма-алгебра $\sigma(\mathcal{A})$ содержится в $m(\mathcal{A})$. Замечаем, что $\mathcal{A} \subseteq m(\mathcal{A}) \subseteq \mathcal{M}$. 
\end{enumerate}
\end{proof}

\section{Измеримое пространство, прямое произведение измеримых пространств, цилиндрические множества (1.2.14 - 1.2.18)}
Обозначение прямого произведения: $\times_{t \in T}\Omega_t$.
\begin{definition}
Пара $(\Omega, \mathcal{F})$, состоящая из некоторого множества $\Omega$ и некоторой $\sigma$-алгебры $\mathcal{F} \subseteq 2^\Omega$, называется \textit{измеримым пространством}. Множества из $\mathcal{F}$ называются \textit{измеримыми множествами}.
\end{definition}

\begin{definition}
Пусть есть множество измеримых пространств $(\Omega_t, \mathcal{F}_t), t \in T$. Тогда прямоугольник $\times_{t \in T}A_t$ называется \textit{измеримым}, если все $A_t \in \mathcal{F}_t$.
\end{definition}

\begin{definition}
Сигма-алгебра, порождённая измеримыми прямоугольниками с $A_t = \Omega_t$ для почти всех $t \in T$, называется \textit{прямым произведением} $\sigma$-алгебр $\mathcal{F}_t$ и обозначается $\otimes_{t \in T}\mathcal{F}_t$
\end{definition}

\begin{definition}
Измеримое пространство $(\times_{t \in T}\Omega_t, \otimes_{t \in T}\mathcal{F}_t)$ называется \textit{прямым произведением} измеримых пространств $(\Omega_t, \mathcal{F}_t), t \in T$.
\end{definition}

\begin{definition}
Для любых $U \subset T, A \subseteq \otimes_{t \in T}\mathcal{F}_t$ множество $C_U(A)$ функций из $\times_{t \in T}\Omega_t$, сужения которых на $U$ принадлежат $A$, называется \textit{цилиндрическим множеством с основанием $A$}. Если $U$ конечное (счётное), то $C_U(A)$ называется цилиндрическим множеством с \textit{конечномерным} (\textit{счётно-конечным}) основанием.
\end{definition}
В адекватных обозначениях для конечномерного основания $U = \{t_1, \ldots, t_n\}$: $C_U(A) = \{\omega \in (\times_{t \in T}\Omega_t \mid (\omega_{t_1}, \ldots, \omega_{t_n}) \in A\}$. 
Важны два факта: 
\begin{enumerate}
    \item Прямое произведение $\times_{t \in T}\Omega_t$ является цилиндрическим множеством.
    \item Дополнение цилиндрического множества суть цилиндрическое множество.
\end{enumerate} 
В обозначениях выше, справедливы следующие теоремы:
\begin{theorem}
Класс $\mathcal{A}$ цилиндрических множеств с конечномерными основаниями является алгеброй и $\sigma(\mathcal{A}) = \otimes_{t \in T}\mathcal{F}_t$.
\end{theorem}
\begin{proof}[Идея Доказательства]
Учитывая два факта выше, для доказательства, что $\mathcal{A}$ - алгебра, достаточно показать, что в $\mathcal{A}$ содержится любое пересечение $C_U(A) \cup C_V(B)$. Берутся два отображения $\pi_1, \pi_2$, сужающие функцию $\omega_t$ на множества $U$ и $V$ соответственно. Дальше доказываются соотношения $C_{U \cup V}(\pi_1^{-1}(A)) = C_U(A), C_{U \cup V}(\pi_2^{-1}(B)) = C_V(B)$, после чего записывается равенство $$C_U(A) \cup C_V(B) = C_{U \cup V}(\pi_1^{-1}(A) \cup (\pi_2^{-1}(B)) \in \mathcal{A}.$$ Вывод второй части теоремы заключается в том, что $\mathcal{A}$ содержится в $\otimes_{t \in T}\mathcal{F}_t$, а кроме того в $\mathcal{A}$ содержится любой прямоугольник $\times_{t \in T}A_t$, в том числе и если взять $A_t = \Omega_t$ для всех $t$, кроме конечного числа. Из этого следует, что $\sigma(\mathcal{A}) = \otimes_{t \in T}\mathcal{F}_t$.
\end{proof}
\begin{theorem}
Пусть множество $B \subseteq \times_{t \in T}\Omega_t$. \newline $B \in \otimes_{t \in T}\mathcal{F}_t \Longleftrightarrow B = C_U(A)$ с конечномерным или счётным основанием $A$.
\end{theorem}
\begin{proof}[Идея Доказательства]
Обозначим $\mathcal{L}$ класс цилиндрических множеств со счётно-конечными или конечномерными основаниями. Доказывается, что $\mathcal{L}$ является $\sigma$-алгеброй. Сложность представляет лишь проверка условия для счётного объединения, которая производится так же, как в доказательстве предыдущей теоремы, вводя сужающие отображения. \newline В силу того, что все цилиндрические множества с конечномерными основаниями лежат в $\mathcal{L}$, верно вложение $\otimes_{t \in T}\mathcal{F}_t \subseteq \mathcal{L}$. Осталось показать, что любое цилиндрическое множество с основанием $U$ лежит в $\otimes_{t \in T}\mathcal{F}_t$. Если основание множества конечное, то утверждение, очевидно, справедливо. Если нет, то вводим класс $\mathcal{L}_U$ множеств $A \in \otimes_{t \in U}\mathcal{F}_t$ таких, что $C_U(A) \in \otimes_{t \in T}\mathcal{F}_t$. Показывается, что $\mathcal{L}_U$ является сигма-алгеброй. Кроме того, класс $\mathcal{L}_U$ содержит все прямоугольники $\times_{t \in U}C_t, C_t \in \mathcal{F}_t$ (это проверяется довольно красиво, вводятся счётномерные прямоугольники $D_m$, в которых первые $m$ сторон равны $C_k, k = 1, \ldots, m$, а остальные возьмём $\Omega_t$. Такие прямоугольники лежат в $\mathcal{L}_U$, а $\times_{t \in U}C_t = \cap_{m = 1}^\infty D_m$). Ну значит $\otimes_{t \in U}\mathcal{F}_t \subseteq \mathcal{L}_U$, что и приводит нас к нужному утверждению.
\end{proof}
\begin{example}
Существуют множества не из прямого произведения сигма-алгебр. Положим $T = [0, 1], \Omega_t = [0, 2], \mathcal{F}_t = \mathcal{B}([0, 2])$. Множество $B = \{\omega \in \times_{t \in T}\Omega_t \mid \sup_{t \in T}\omega_t = 1\}$ не лежит в $\otimes_{t \in T}\mathcal{F}_t$. Доказываем от противного, применяя предыдущую теорему. По ней основание $A$ цилиндрического множества $B$ должно быть прямым произведением счётного числа $\mathcal{F}_{t_n}$. Поскольку мощность $T$ - континуум, можно изменить функцию $\omega \in B$, положив её равной 2 во всех точках $[0, 1] \backslash \{t_n\}$. Тогда её принадлежность $B$ не изменится, но супремум уже будет равен 2, что приводит к противоречию.
\end{example}

\section{Понятие случайного процесса. Теорема Колмогорова о существовании случайного
процесса с данными конечномерными распределениями (3.1.1 – 3.1.4)}
Мы живём в некотором вероятностном пространстве $(\Omega, \mathcal{F}, P)$. Надо понимать, что у Круглова не случайный вектор является набором случайных величин, а случайная величина является одномерным случайным вектором. А случайный вектор - это измеримое отображение $\Omega \rightarrow R^d$. Измеримая функция же в свою очередь определяется точно так же, как мы привыкли, только вместо $\mathcal{B}(R)$ берётся $\mathcal{B}(R^d)$.
\begin{definition}
Произвольное семейство случайных векторов $X_t: \Omega \rightarrow R^d, t \in T$ называется \textit{случайным процессом}.
\end{definition}
\begin{definition}
\textit{Конечномерным распределением} случайного процесса называется мера $P_{t_1, \ldots, t_n}\{A\} = P\{(X_{t_1}, \ldots, X_{t_n}) \in A\}, A \in \mathcal{B}(R^{dn})$.
\end{definition}
Семейство конечномерных распределений является основной характеристикой случайного процесса. Дальше в билете формулируем две теоремы о них: простую и фундаментальную(ну ясен пень, это ж Колмогоров! Кто-нибудь видел не оч важную теорему Колмогорова?). Фундаментальность заключается в том, что она, по сути, гласит, что по заданным конечномерным распределениям можно построить случайный процесс.
\begin{theorem}
Конечномерные распределения удовлетворяют условиям согласованности:
$$P_{t_1, \ldots, t_n}\{\times_{k=1}^n A_k\} = P_{t_{\pi(1)}, \ldots, t_{\pi(n)}}\{\times_{k=1}^n A_{\pi(k)}\}$$

$$P_{t_1, \ldots, t_{n+1}}\{\times_{k=1}^n A_k \times R^d\} = P_{t_1, \ldots, t_n}\{\times_{k=1}^n A_k\}$$, где, очевидно, $t_i \in T$ - любые, $A_i \in \mathcal{B}(R^d)$, $\pi$ - перестановка.
\end{theorem}
\begin{proof}[Идея доказательства]
Идеи нет, две строчки, тупо по определению. Не боимся, следующее доказательство отыграется.
\end{proof}
\begin{theorem}[Колмогоров]
Пусть семейство вероятностей $P_{t_1, \ldots, t_n}\{A\}, A \in \mathcal{B}(R^{dn})$ удовлетворяет условиям согласованности из предыдущей теоремы. Тогда существуют вероятность $P^T: \mathcal{B}((R^d)^T) \rightarrow [0, 1]$ и случайный процесс $X = {X_t, t \in T}$, определённый на вероятностном пространстве $((R^d)^T, \mathcal{B}((R^d)^T), P^T)$, такие, что $$P_{t_1, \ldots, t_n}^T\{A\} = P_{t_1, \ldots, t_n}\{A\}, \forall t_i \in T, A \in \mathcal{B}(R^{dn})$$.
\end{theorem}
\begin{proof}[Идея доказательства]
Оно обосраться какое здоровое, страницы 3, но по сути строится мера $\mu\{C_U(A)\} = P_{t_1, \ldots, t_n}\{A\}$ на алгебре цилиндрических множеств с конечномерными основаниями $A$, проверяется, что это действительно мера, и потом строится искомая $P^T$ как продолжение этой меры. Построив вероятность, процесс строится тупо полагая $X_t(\omega) = \omega_t, \forall \omega \in (R^d)^T, t \in T$.
\end{proof}

\section{Эквивалентные, неотличимые, одинаково распределенные, непрерывные случайные процессы (3.1.6 - 3.1.12)}
\begin{definition}
Случайные процессы $\{X_t, t \in T\}$ на $(\Omega, \mathcal{F}, P)$ и $\{X'_t, t \in T\}$ на $(\Omega', \mathcal{F}', P')$ называются \textit{одинаково распределёнными}, если для любых $t_1, \ldots, t_n, A$ выполнено $$P\{(X_{t_1}, \ldots, X_{t_n}) \in A\} = P'\{(X'_{t_1}, \ldots, X'_{t_n}) \in A\}$$.
\end{definition}
\begin{definition}
Пусть случайные процессы $X_t$, $X'_t$ определены на одном вероятностном пространстве и принимают значения в $R^d$. Если $\forall t \in T : P\{X_t \neq X'_t\} = 0$, то эти процессы называются \textit{эквивалентными}. Эквивалентные случайные процессы называются \textit{версиями} друг друга. 
\end{definition}
Если интерпретировать $T$ как время, то эквивалентность означает равенство почти наверное в любой фиксированный момент времени. Понятно, что эквивалентные процессы одинаково распределены.
\begin{definition}
Пусть случайные процессы $X_t$, $X'_t$ определены на одном вероятностном пространстве и принимают значения в $R^d$. Пусть есть некое $\Omega'$ такое, что $P\{\Omega'\} = 1$ и $\forall \omega \in \Omega'$ совпадают траектории $X_t(\omega)$ и $X'_t(\omega)$. Такие случайные процессы называются \textit{неотличимыми}.
\end{definition}
Неотличимые случайные процессы эквивалентны. Неотличимость -- самое сильное из возможных свойство двух процессов, далее эквивалентность и только потом одинаковая распределённость. Однако, если потребовать некоторые дополнительные условия на процессы и/или множество $T$, то можно показать, что и из эквивалентности следует неотличимость. Этому посвящены следующие две теоремы.
\begin{theorem}
Эквивалентные процессы со счётным множеством $T$ неотличимы.
\end{theorem}
\begin{proof}[Идея доказательства]
В качестве $\Omega'$ из определения неотличимых процессов возьмём $\cap_{t \in T}\{X_t = X'_t\} \in \mathcal{F}$.
\end{proof}
Для следующей теоремы понадобится ещё одно
\begin{definition}
Случайный процесс называется (\textit{почти})\textit{непрерывным} (\textit{непрерывным слева/справа}), если (почти) все его траектории непрерывны (непрерывны слева/справа).
\end{definition}
\begin{theorem}
Если эквивалентные процессы почти всюду непрерывны слева/справа, а множество $T$ выпукло, то они неотличимы.
\end{theorem}
\begin{proof}[Идея доказательства]
Берём $\Omega''$ как в предыдущей теореме, только на $T \cap Q$, где $Q$ - множество рациональных чисел. Для $\Omega''$ всё хорошо. Все остальные t приближаем последовательностью $\{t_n\}, t_i \in T \cap Q$. 
\end{proof}

\section{Стохастически непрерывные случайные процессы (3.1.13 - 3.1.14)}
Я очень сильно подозреваю, что $t^*$ и $t_*$ - это супремум и инфимум $T$ соответственно. Очень надеюсь, что это правда.
\begin{definition}
Случайный процесс $X = {X_t, t \in T}$ с выпуклым множеством $T$ называется \textit{стохастически непрерывным слева}, если $$\lim_{t \uparrow s}P\{\|X_t - X_s\| > \epsilon\} = 0, \forall \epsilon > 0, s > t_*$$
\textit{Стохастическая непрерывность справа}: $$\lim_{t \downarrow s}P\{\|X_t - X_s\| > \epsilon\} = 0, \forall \epsilon > 0, s < t^*$$
Случайный процесс называется \textit{стохастически непрерывным}, если он стохастически непрерывен слева и справа.
\end{definition}
\begin{definition}
Случайный процесс называется \textit{равномерно стохастически непрерывным}, если $$\lim_{h \rightarrow 0}\sup_{|t - s| < h}P\{\|X_t - X_s\| > \epsilon\} = 0, \forall \epsilon > 0$$
\end{definition}
\begin{theorem}
Стохастически непрерывный случайный процесс равномерно стохастически непрерывен.
\end{theorem}
\begin{proof}[Идея доказательства]
Стохастическая непрерывность: $\lim_{t \rightarrow s}P\{\|X_t - X_s\| > \epsilon\} = 0, \forall \epsilon > 0, s \in T$.
Доказываем от противного, отсутствие равномерности означает, что $\sup_{\|t - s\| < h_n}P\{\|X_t - X_s\| > \epsilon\} > \alpha$, для некоторых $\epsilon, \alpha > 0$ и монотонно убывающей последовательности $\{h_n\}, h_n \downarrow 0$. Берём подпоследовательности $t_n, s_n$, сходящиеся к некоторому числу $s$, такие, чтобы для них было верно $P\{\|X_{t_n} - X_{s_n}\| > \epsilon\} > \frac{\alpha}{2}$. Противоречие возникает, если устремить $n \rightarrow \infty$ в неравенстве $$P\{\|X_{t_n} - X_{s_n}\| > \epsilon\} \le P\{\|X_{t_n} - X_{s}\| > \frac{\epsilon}{2}\} + P\{\|X_{s_n} - X_{s}\| > \frac{\epsilon}{2}\}$$
\end{proof}

\section{Теорема существования сепарабельных случайных процессов (3.2.1 - 3.2.5)}
Рассматривается задача: вычислить вероятность того, что траектории случайного процесса лежат в данном множестве. Фишка в том, что эта задача может оказаться некорректной для некоторого процесса $X$ (например для $X_t(\omega) = 1$ при $\omega \in A$, иначе $0$, для неборелевского $A \subset [0, 1]$). С другой стороны, можно взять процесс $Y$, эквивалентный $X$ (в примере выше просто $Y_t = 0$), для которого задача будет звучать корректно. Это утверждение верно для любого $X$ и является теоремой, которую мы сформулируем дальше (доказана Дубом. Даже Дуб что-то может доказать, а ты нет...). Процесс $Y$ называют \textit{сепарабельной версией} $X$. Определим теперь всё более формально.

Кроме уже надоевшего процесса $X$, определённого на вероятностном пространстве $(\Omega, \mathcal{F}, P)$ со значениями в $R^d$, введём ещё обозначение $\overline{X(U, \omega)}$. Это замыкание множества $X(U, \omega) = \{X_t(\omega) \mid t \in U\}$. Кроме того, назовём \textit{относительным интервалом} пересечение $(a, b) \cap T$ вещественного интервала $(a, b)$ и множества $T$.

\begin{definition}
Случайный процесс называется \textit{сепарабельным}, если существует конечное или счётное $S \subset T$ и событие $N \in \mathcal{F}, P\{N\} = 0$ такие, что $$X_u(\omega) \in \cap_{J: u \in J} \overline{X(J \cap S, \omega)}$$ для любых $u \in T$ и $\omega \notin N$, а $J$ - относительный интервал, содержащий $u$. Пересекаем по всем таким $J$. $S$ называется \textit{сепарантой}, а $N$ - \textit{исключительным событием} случайного процесса.
\end{definition}
Смысл условия в определении заключается в том, что почти каждая траектория случайного процесса определяется своими значениями на счётном множестве $S$. Понятно, что процесс может быть несепарабельным только в случае несчётного $T$, иначе можно просто взять $S = T$. Верно также утверждение, что почти всюду непрерывный слева/справа процесс с выпуклым $T$ является сепарабельным.
Следующая теорема является критерием сепарабельности случайного процесса.
\begin{theorem}
Случайный процесс сепарабельный $\Longleftrightarrow$ существуют счётное множество $S$, событие нулевой вероятности $N$ такие, что для любого замкнутого множества $K \subset \overline{R}^d$ и относительного интервала $J$ $$\cap_{t \in J \cap S}\{X_t \in K\} \subseteq \cap_{t \in J}\{X_t \in K\} \cup N$$ или (эквивалентное условие) $$\cap_{t \in J \cap S}\{X_t \in K\} \subseteq \{X_u \in K\} \cup N, \forall u \in J$$
\end{theorem}
\begin{proof}[Идея доказательства]
Доказывается цепочка следствий: условие в определении сепарабельного процесса $\Longrightarrow$ первое утверждение теоремы $\Longrightarrow$ второе утверждение теоремы $\Longrightarrow$ условие в определении сепарабельного процесса. Каждое из них несложно и разбирается непосредственно при запоминании доказательства.
\end{proof}
\begin{theorem}
Для любого случайного процесса найдутся счётное множество $S \subset T$ и события $N^u \in \mathcal{F}, u \in T$ нулевой вероятности такие, что $$X_u(\omega) \in \cap_{J: u \in J} \overline{X(J \cap S, \omega)}$$ для любых $u \in T$ и $\omega \notin N^u$.
\end{theorem}
\begin{proof}[Идея доказательства]
Рассуждая как в доказательстве предыдущей теоремы, получим, что утверждение в этой теореме равносильно $$\cap_{t \in J \cap S}\{X_t \in K\} \subseteq \{X_u \in K\} \cup N^u$$ Дальше какие-то нее*ические финты ушами для доказательства этого утверждения.
\end{proof}
Теперь самое время для, собственно, основной теоремы. Предварительно только определим \textit{сепарабельную версию} процесса $X$, как некий эквивалентный процесс $Y$, обладающий свойством сепарабельности.
\begin{theorem}
Любой случайный процесс $X$ имеет сепарабельную версию $Y$, $Y_t:\Omega \rightarrow \overline{R}^d$.
\end{theorem}
\begin{proof}[Идея доказательства]
Берём $S, N^u$ из предыдущей теоремы. Обозначаем $M^u$ множество всех тех $\omega$, для которых не выполняется утверждение в предыдущей теореме. Очевидно, $M^u \in N^u, P\{M^u\} = 0$. Вводим процесс $Y = \{Y_u, u \in T\}$, где $Y_u(\omega)$ совпадает с $X_u(\omega)$ везде, кроме случая, когда $u \notin S, \omega \in M^u$. В этом случае берём $Y_u(\omega) = x$, где $x$ - любая точка множества $\cap_{J: u \in J} \overline{X(J \cap S, \omega)}$ (хотя бы одна существует, это доказывается в первой главе книги).
\end{proof}
Последнее утверждение касается сепарант сепарабельных стохастически непрерывных процессов с выпуклым множеством $T$.
\begin{theorem}
Если процесс сепарабельный и стохастически непрерывный с выпуклым множеством $T$, то в качестве сепаранты можно взять любое всюду плотное множество счётное множество $S \subset T$.
\end{theorem}
\begin{proof}[Идея доказательства]
Берём $N, S_0$ - некоторые исключительное событие и сепаранта, $S$ - произвольное всюду плотное множество. Приближаем $u$ последовательностью $\{s_n\}, s_n \in S$, чтобы ещё и $\|X_{s_n} -X_u\|$ сходилось к $0$ по вероятности. Можно считать, что последовательность сходится тогда почти всюду(иначе выберем подпоследовательность, сходящуюся почти всюду). Вводим событие $N^u = \Omega \backslash \{\lim_{n \rightarrow \infty}\|X_{s_n} - X_u\| = 0\}$. Оно нулевой вероятности. Вводим новое событие $N_0 = N \cup \cup_{u \in S_0}N^u$. Оно тоже нулевой вероятности, а дальше показывается, что оно и будет исключительным событием для сепаранты $S$.
\end{proof}

\section{Свойства вещественных сепарабельных процессов (3.2.6)}
Обозначим $J(T)$ - класс всех относительных интервалов множества $T \subseteq R$, $\mathcal{E}$ - класс множеств вида $[-\infty, a], [a, b], [a, \infty], a \le b, a,b \in R$. В формулировке следующей теоремы выбираются любые множества: $J \in J(T), K \in \mathcal{E}, u \in J, \omega \notin N$, где $N$ - исключительное событие. 
\begin{theorem}
Пусть всё так же есть вещественный случайный процесс $X$, который ещё и сепарабелен с сепарантой $S$ и исключительным событием $N$. Тогда следующие утверждения эквивалентны:
\begin{enumerate}
    \item $\cap_{t \in J \cap S}\{X_t \in K\} \subseteq \cap_{t \in J}\{X_t \in K\} \cup N$
    \item $\inf_{t \in J}X_t(\omega) = \inf_{t \in J \cap S}X_t(\omega)$; \newline $\sup_{t \in J}X_t(\omega) = \sup_{t \in J \cap S}X_t(\omega)$
    \item $\inf_{t \in J \cap S}X_t(\omega) \le X_u(\omega) \le \sup_{t \in J \cap S}X_t(\omega)$
    \item $\lim_{J \ni t \rightarrow u} \inf X_t(\omega) = \lim_{J \cap S \ni t \rightarrow u} \inf X_t(\omega)$; \newline $\lim_{J \ni t \rightarrow u} \sup X_t(\omega) = \lim_{J \cap S \ni t \rightarrow u} \sup X_t(\omega)$
    \item $\lim_{J \cap S \ni t \rightarrow u} \inf X_t(\omega) \le X_u(\omega) \le \lim_{J \cap S \ni t \rightarrow u} \sup X_t(\omega)$
\end{enumerate}
\end{theorem}
Эти пять условий на самом деле довольно просто запоминаются мнемонически, достаточно помнить, что 1 - это просто из теоремы предыдущего билета, 2 - два одинаковых утверждения для точных нижних и верхних граней, меняется (сужается) лишь множество, в котором мы смотрим $t$, третье - тупо неравенство для инфимума и супремума из второго, а 4 и 5 - по сути 2 и 3, только с добавленными пределами при $t \rightarrow u$.
\begin{proof}[Идея доказательства]
Доказываем цепочку следствий: $1 \Longrightarrow 2 \Longrightarrow 3 \Longrightarrow 4 \Longrightarrow 5 \Longrightarrow 1$. По очереди:
\begin{enumerate}
    \item От противного
    \item $\inf_{t \in J}X_t(\omega) \le X_u(\omega) \le \sup_{t \in J}X_t(\omega)$ и применяем 2.
    \item Положим $J_\epsilon(u) = (u - \epsilon, u + \epsilon) \cap T$. Тогда в силу 3 верно, что $\forall v \in J_\epsilon(u): \inf_{t \in J_\epsilon(u) \cap S}X_t(\omega) \le X_v(\omega)$, откуда $\inf_{t \in J_\epsilon(u) \cap S}X_t(\omega) \le \inf_{v \in J_\epsilon(u)}X_v(\omega)$. Дальше устремляем $\epsilon \rightarrow 0$. Аналогично доказываем второе равенство.
    \item Очевидно, поскольку $$\lim_{J \ni t \rightarrow u} \inf X_t(\omega) \le X_u(\omega) \le \lim_{J \ni t \rightarrow u} \sup X_t(\omega)$$, а потом применяем 4.
    \item От противного. В предположении противного, найдётся некоторое $K \in \mathcal{E}$, которое всё нам портит. Разберём все случаи таких $K$ (три вида возможных множеств в $\mathcal{E}$). В каждом из них конечный край интервала будет отделять $X_u(\omega)$ и предел соответствующих точных граней, что невозможно.
\end{enumerate}
\end{proof}

\section{Достаточные условия непрерывности случайных процессов (3.3.1 - 3.3.4)}
\begin{theorem}
Сепарабельный случайный процесс с выпуклым множеством $T$ почти всюду непрерывен, если $$\lim_{h \downarrow 0}P\{\sup_{|s - t| < h} \|X_s - X_t\| > \epsilon\} = 0, \forall \epsilon > 0$$.
\end{theorem}
\begin{proof}[Идея доказательства]
То, что должно быть больше эпсилона, -- убывающая последовательность $Z_h$. Тогда есть её предел. В силу условия теоремы, этот предел $Z$ равен нулю почти всюду. Тогда траектории непрерывны в любом $\omega \in Z$.
\end{proof}
\begin{theorem}
Сепарабельный случайный процесс с $T = [a, b]$ почти всюду непрерывен $\Longleftrightarrow$ выполнено условие из предыдущей теоремы.
\end{theorem}
\begin{proof}[Идея доказательства]
По предыдущей теореме следствие в одну сторону верно. В другую пользуемся теоремой Кантора, по которой из непрерывности почти всюду следует равномерная непрерывность почти всюду, а из сходимости почти всюду вытекает сходимость по вероятности.
\end{proof}
Разница между первой и следующей теоремой в том, что в первой мы берём любые $s, t \in T$ внутри вероятности, а в этой смотрим супремум по $s$ вероятностей, где $s$ уже фиксированное.
\begin{theorem}
Сепарабельный случайный процесс с выпуклым множеством $T$ почти всюду непрерывен, если $$\lim_{h \downarrow 0}\frac{1}{h}\sup_{s \in T}P\{\sup_{|s - t| < h} \|X_s - X_t\| > \epsilon\} = 0, \forall \epsilon > 0$$
\end{theorem}
\begin{proof}[Идея доказательства]
Сначала предположим, что $T = [a, b]$. Можно нормировать процесс и считать, что $[a, b] = [0, 1]$. Показываем выполнение условия из первой теоремы. В общем случае представим выпуклое множество как объединение счётного числа вложенных сегментов. На каждом из них мы доказали, что есть событие единичной вероятности такое, что для любого исхода из этого события траектория непрерывна. Возьмём желаемое событие для всего множества $T$ как пересечение найденных событий на каждом сегменте.
\end{proof}

Последняя теорема - какой-то пи**ец, запомнить можно, доказывать не советую. Выглядит полезно и классно, и оценка скорости равномерной сходимости, и все дела, но это всё вплоть до доказательства.
\begin{theorem}
Пусть случайный процесс с выпуклым множеством $T$ удовлетворяет условию: $$P\{\|X_s - X_t\| \ge p(|t - s|)\} \le q(|t - s|)$$ для любых $s, t \in T : |t - s| < \delta, \delta > 0$ и для некоторых функций $p,q: [0, \delta] \rightarrow R_+$ таких, что $$\int_0^\delta \frac{p(u)}{u} du < \infty, \int_0^\delta \frac{q(u)}{u^2} du < \infty.$$ Тогда существует непрерывная версия $Y$ процесса $X$. Более того, для любого конечного отрезка $[a, b] \subseteq T$ существует функция $H: \Omega \rightarrow \{1, 2, \ldots\}$ такая, что $\forall \omega \in \Omega, h \in (0, 2^{-H(\omega)} \wedge \delta)$ выполнено неравенство $$\sup_{s, t \in [a, b]: |t - s| < \frac{h}{2}} \|Y_s(\omega) - Y_t(\omega)\| \le \frac{2}{\ln 2} \int_0^h \frac{p(u)}{u} du$$
\end{theorem}
\begin{proof}[Идея доказательства]
Эта жопа разбивается на 5 пунктов, каждый из которых всё больше обобщает теорему.
\begin{enumerate}
    \item Сначала снова полагаем, что $T = [a, b]$, дополнительно можно положить $\delta = 1$. Доказывается оценка сверху на $\|X_s(\omega) - X_t(\omega)\|$ на некотором множестве (получается константа, умноженная на первый из интегралов в условии).
    \item Доказывается, что существует предел $\lim_{S \ni s \rightarrow t}X_s(\omega)$. Для этого берётся $S \ni s_n \rightarrow t$ и подставляется вместо $X_s, X_t$ в неравенство из первого пункта доказательства. Получается сходимость $\{X_{s_n}(\omega)\}$ в силу условия Коши. Кроме того, вводится искомый новый процесс $Y$.
    \item Доказывается эквивалентность процессов $X$ и $Y$.
    \item Доказывается неравенство в условии теоремы. Разбираются случаи разных $\omega$, в каждом приходим к нужному результату, а кроме того показываем непрерывность $Y$.
    \item Доказываем в общем случае для любого выпуклого $T$. Опять представляем выпуклое множество как объединение счётного числа сегментов, строим искомый процесс $Y$ через процессы на каждом сегменте, которые существуют по доказанному выше.
\end{enumerate}
\end{proof}

\section{Теорема Колмогорова о непрерывных случайных процессах (3.3.5)}
Какая-то странная теорема Колмогорова... Во-первых потому, что Круглов не упомянул в учебнике, что это именно Колмогоров, во-вторых почему-то она не очень фундаментально звучит, в-третьих, всего-то страница доказательства, а я с трудом верю, что Колмогоров называл своим именем то, что Тыртышникову очевидно. Ну да ладно, собственно, сама
\begin{theorem}[Колмогоров]
Если случайный процесс с выпуклым множеством $T$ удовлетворяет условию $$\mathsf{E} \|X_t - X_s\|^\alpha \le c|s - t|^{1 + \beta}$$ для любых $s, t$ и каких-то положительных $a, b, c$, то он имеет непрерывную версию $Y$ со свойством $$\lim_{h \downarrow 0}\frac{1}{h^\gamma}\sup_{s, t \in [a, b]: |t - s| < h} \|Y_t - Y_s\| = 0$$ для любого $\gamma \in (0, \frac{\beta}{\alpha})$ и любого сегмента $[a, b] \in T$.
\end{theorem}
\begin{proof}[Идея доказательства]
Выбирается последовательность $\{\delta_n\}, \delta_n \in (0, \frac{\beta}{\alpha}), \delta_n \rightarrow \frac{\beta}{\alpha}$. Выбираем $p_n(u) = u^{\delta_n}, q_n(u) = cu^{1 + (\beta - \alpha\delta_n)}, u \in [0, 1]$. Для них применима последняя теорема из предыдущего билета. Для каждого $n$ есть непрерывная версия $Y^{(n)}$. Показывается, что на некотором событии единичной вероятности выполняется $Y_t^{(n)} = Y_t^{(n + 1)}$ для любого $t \in T$. Вводится новый процесс $Y$, равный $Y^{(1)}$ на пересечении всех таких событий единичной вероятности, а в остальных точках - любой. Предел для супремума получается из неравенства в последней теореме из предыдущего билета.
\end{proof}

\section{Функции без разрывов второго рода (3.4.1 - 3.4.5)}
Множество $T$ полагается бесконечным и выпуклым.
\begin{definition}
Функция $f: T \rightarrow R^d$ \textit{не имеет разрывов второго рода}, если есть $f(t-) \forall t > t_*$ и $f(t+) \forall t < t^*$.
\end{definition}
Доопределим $f(t_*-) = f(t_*), f(t^*+)=f(t^*)$. Разность $\Delta f(t) = f(t+) - f(t-)$ назовём \textit{скачком} функции $f$ в точке $t \in T$, а $\|\Delta f(t)\|$ - \textit{величиной скачка}.
\begin{theorem}
Пусть дана функция без разрывов второго рода. Тогда $\forall c > 0, a, b \in T, a < b$ множество $E_{c, a, b} = \{t \in [a, b] \mid \|\Delta f(t)\| \ge c\}$ конечно, а множество $E = \{t \in [a, b] \mid \|\Delta f(t)\| \ge 0\}$ конечно или счётно.
\end{theorem}
\begin{proof}[Идея доказательства]
Выпуклое множество $T$ опять представляем как объединение счётного числа расширяющихся интервалов, множество $E$ как объединение по всем этим интервалам соответствующих неотрицательных норм. К нему ещё добавим те точки из $t_*, t^*$, которые лежат в $T$, и в которых скачки положительны. На каждом таком сегменте $\{t \in [a_n, b_n] \mid \|\Delta f(t)\| \ge 0\} = \cup_{r = 1}^\infty E_{\frac{1}{r}, a_n, b_n}$, поэтому остаётся доказать конечность множества $E_{c, a, b}$. Доказываем от противного, предполагая бесконечность множества $E_{c, a, b}$ и выбирая монотонную возрастающую сходящуюся подпоследовательность.
\end{proof}
\begin{theorem}
Если нет разрывов второго рода, то $\sup_{a \le t \le b} \|f(t)\| \le \infty, \forall a, b \in T$. Если, ко всему прочему, ещё и $\|\Delta f(t)\| \le c$ для некоторого $c > 0$ и всех $t$, то $\forall \epsilon > 0, a, b$ $\exists \delta(\epsilon, a, b) > 0$ такое, что $\|f(t) - f(s)\| < c + \epsilon$ при любых $s, t \in [a, b], |t - s| < \delta$.
\end{theorem}
По сути утверждается, что точная верхняя грань конечна, а если скачки ограничены какой-то константой, то на любом сегменте для наперёд заданного эпсилона можно выбрать дельту так, чтобы функция за время, не большее, чем дельта, (всё ещё интерпретируем $T$ как время) изменилась меньше, чем на ограничивающую константу плюс эпсилон.
\begin{proof}[Идея доказательства]
Первое утверждение совсем простое, от противного, надо взять всё ту же монотонную возрастающую последовательность, чтобы $\|f(t_n)\| > n$, тогда её предел бесконечен, но равен $\|f(t-)\|$, где $\lim_{n \rightarrow \infty} t_n = t$.

Второе утверждение тоже доказываем от противного, записывая отрицание утверждения, выбирая $\delta_n \downarrow 0$ и $s_n, t_n : |s_n - t_n| < \delta_n, \|f(t_n) - f(s_n)\| \ge c + \epsilon$. Можно считать, что $s_n, t_n$ сходятся к одному числу $t$. Переходя к пределу, получаем противоречие.
\end{proof}
\begin{definition}
Функция называется \textit{регулярной справа(слева)}, если она непрерывна справа(слева) в каждой точке $t \in T, t < t^* (t > t_*)$ и имеет предел слева(справа) в каждой точке $t \in T, t > t_* (t < t^*)$. 
\end{definition}
\begin{theorem}
Пусть функция определена на всюду плотном подмножестве $S$ множества $T$. Предположим, что существуют $\lim_{S \ni s \uparrow t} f(s) = g(t-) \in R^d, \forall t \in T, t > t_*$ и $\lim_{S \ni s \downarrow t} f(s) = g(t+) \in R^d, \forall t \in T, t < t^*$. Тогда функция $g(t-), t \in T, t > t_*$ регулярна слева, функция $g(t+), t \in T, t < t^*$ регулярна справа и $\sup_{t \in [a, b] \cap S} \|f(t)\| < \infty, \forall a, b \in T, a < b$.
\end{theorem}
\begin{proof}[Идея доказательства]
Докажем, например, для $g(t+)$. Сначала показывается, что функция непрерывна, после чего записываются неравенства, из которых следует условие Коши (берём точку $a \in (t_*, t^*), \lim_{S \ni s \uparrow a}f(s) = g(a-)$, в её левой окрестности, в которой функция отличается не более, чем на $\epsilon$, берём две точки $t', t''$, находим последовательности, к ним сходящиеся и записываем разность $\|g(t'+) - g(t''+)\|$ через предел этих последовательностей).
\end{proof}

\section{Случайные процессы без разрывов второго рода: знать формулировки теорем (3.4.7
- 3.4.8)}
Спасибо Круглову за отсутствие необходимости знать доказательства, в противном случае можно было бы сразу идти на пересдачу. 
\begin{theorem}
Пусть случайный процесс с выпуклым множеством $T, t_* \in T$ удовлетворяет условию $$P\{\|X_{t_1} - X_{t_2}\|\|X_{t_2} - X_{t_3}\| \ge p(t_3 - t_1)\} \le q(t_3 - t_1)$$ для любых $t_i \in T, t_1 < t_2 < t_3, t_3 - t_1 < \delta, \delta > 0$ и для некоторых неубывающих функций $p,q: [0, \delta] \rightarrow R_+$ таких, что $$\int_0^\delta \frac{p(u)}{u} du < \infty, \int_0^\delta \frac{q(u)}{u^2} du < \infty.$$ Тогда процесс $X$ имеет версию $Y$ без разрывов второго рода.
\end{theorem}
Теорема выше является обобщённым случаем второй теоремы.
\begin{theorem}[Ченцов]
Пусть случайный процесс с выпуклым множеством $T, t_* \in T$ удовлетворяет условию $$\mathsf{E}(\|X_{t_1} - X_{t_2}\|\|X_{t_2} - X_{t_3}\|)^\alpha \le c|t_3 - t_1|^{1 + \beta}$$ для некоторых $\alpha, \beta > 0, t_i \in T, t_1 < t_2 < t_3$. Тогда процесс $X$ имеет версию $Y$ без разрывов второго рода. 
\end{theorem}
В некотором роде эти две теоремы являются аналогами теорем 8.4 и 9.1(Колмогорова), только там непрерывные, а здесь без разрывов второго рода.

\

\section{Фильтрации и их свойства, естественные фильтрации случайных процессов (3.5.1
- 3.5.6)}
\begin{definition}
Семейство $\mathrm{F}_T = \{\mathcal{F}_t \mid \mathcal{F}_t \subseteq \mathcal{F}, t \in T\}$ сигма-алгебр называется \textit{фильтрацией}, если $\mathcal{F}_s \subseteq \mathcal{F}_t$ для любых $s < t, s, t \in T$.
\end{definition}
\begin{definition}
Фильтрация называется \textit{расширенной}, если любое множество $A \in \mathcal{F}$ нулевой вероятности принадлежит всем сигма-алгебрам $\mathcal{F}_t$.
\end{definition}

Введём обозначения $\mathcal{F}_{t+} = \cap_{s > t} \mathcal{F}_s, \mathcal{F}_{t-} = \sigma(\mathcal{F}_s, s < t)$.
\begin{definition}
Фильтрация называется \textit{непрерывной справа(слева)}, если $\mathcal{F}_t = \mathcal{F}_{t+} (\mathcal{F}_{t-} = \mathcal{F}_t), \forall t \in T$.
\end{definition}
Фильтрация со счётным $T$ непрерывна слева $\Longleftrightarrow$ все сигма-алгебры равны между собой.

Пусть дана фильтрация $\mathrm{F}_T$ с выпуклым параметрическим множеством $T$. Введём следующие классы: $\mathcal{N} = \{A \mid A \in \mathcal{F}, P(A) = 0\}, \mathcal{G}_t = \sigma(\mathcal{F}_t, \mathcal{N}), \mathcal{G}_{t+} = \cap_{s > t} \mathcal{G}_s, \forall t < t^*, \mathcal{G}_{t^*+} = \mathcal{G}_{t^*}$.
\begin{theorem}
В обозначениях выше, для определённой выше фильтрации $\mathrm{F}_T$ справедливо, что фильтрация $\mathcal{G}_{T+}$ непрерывна справа, расширена и обладает минимальным свойством в том смысле, что $\mathcal{G}_{t+} \subseteq \mathcal{H}_t, \forall t \in T$ для любой расширенной, непрерывной справа фильтрации $\{\mathcal{H}_t \mid \mathcal{F}_t \subseteq \mathcal{H}_t, \forall t \in T\}$.
\end{theorem}
\begin{proof}[Идея доказательства]
Коротенькое, почти очевидное доказательство. Расширенность очевидна из определения, непрерывность просто расписывается по определению, минимальность выводится из того, что $\mathcal{G}_t \subseteq \sigma(\mathcal{H}_t, \mathcal{N})$, так как $\mathcal{F}_t \subseteq \mathcal{H}_t$.
\end{proof}
\begin{definition}
Случайный процесс называется \textit{согласованным} с фильтрацией, если для любого $t$ случайный вектор $X_t$ измерим относительно $\mathcal{F}_t$.
\end{definition}
\begin{definition}
Фильтрация $\mathcal{F}_T^{(X)}, \mathcal{F}_t^{(X)} = \sigma(X_s, s \le t)$ называется \textit{естественной фильтрацией} случайного процесса $X$.
\end{definition}
\begin{definition}
Фильтрация $\mathcal{G}_T^{(X)}, \mathcal{G}_t^{(X)} = \sigma(\mathcal{F}_t^{(X)}, \mathcal{N})$ называется \textit{расширенной естественной фильтрацией} случайного процесса $X$.
\end{definition}
Случайный процесс согласован с каждой из своих естественных фильтраций. Если вспомнить, что мы привыкли интерпретировать $T$ как время, то $\mathcal{F}_t^{(X)}$ содержит информацию о поведении процесса вплоть до времени $t \in T$.
\begin{theorem}
Если случайный процесс с выпуклым множеством $T$ непрерывен слева, то его естественные фильтрации $\mathcal{F}_T^{(X)}, \mathcal{G}_T^{(X)}$ тоже непрерывны слева.
\end{theorem}
\begin{proof}[Идея доказательства]
Доказываем для $\mathcal{F}_T^{(X)}$. Берём последовательность $t_n < t, t_n \uparrow t$. Случайный вектор $X_{t_n}$ измерим относительно $\mathcal{F}_{t_n}^{(X)}$, значит и относительно $\mathcal{F}_{t-}^{(X)}$. Переходим к пределу, получаем, что $\mathcal{F}_t^{(X)} \subseteq \mathcal{F}_{t-}^{(X)}$, а поскольку обратное вложение тоже верно, получаем равенство. 
\end{proof}
\begin{theorem}
Если случайный процесс с выпуклым множеством $T$ стохастически непрерывен слева, то его расширенная естественная фильтрация $\mathcal{G}_T^{(X)}$ тоже непрерывна слева.
\end{theorem}
\begin{proof}[Идея доказательства]
Опять берём $t_n < t, t_n \uparrow t$, тогда $\{X_{t_n}\}$ сходится по вероятности к $X_t$. Можем считать, что почти всюду. Берём событие $\Omega_t$ из тех исходов, где сходится. Оно единичной вероятности, значит лежит во всех $\mathcal{G}_{s-}^{(X)}$. Вводим $Y_{t_n} = \mathbf{1}_{\Omega_t}X_{t_n}, Y_t = \mathbf{1}_{\Omega_t}X_t$. Из измеримости $Y_{t_n}$ относительно $\mathcal{G}_{t-}^{(X)}$ следует, что $Y_t$ тоже. Убеждаемся, что $X_t$ тогда тоже измерим, откуда $\mathcal{G}_{t}^{(X)} \subseteq \mathcal{G}_{t-}^{(X)}$. Обратное вложение тоже верно, значит есть равенство.
\end{proof}

\section{Марковские моменты (3.6.1 - 3.6.7)}
\begin{definition}
Функция $\tau: \Omega \rightarrow T \cup \{\infty\}$ называется \textit{$\mathrm{F}_T$-марковским моментом} или \textit{марковским моментом относительно фильтрации $\mathrm{F}_T$}, если $\{\tau \le t\} \in \mathcal{F}_t, \forall t \in T$.
\end{definition}
\begin{example}
Функция, тождественная равная $u \in T$, является марковским моментом, потому что множество $\{\tau \le t\}$ либо $\emptyset$, либо $\Omega$, а оба они лежат в любой сигма-алгебре.
\end{example}
\begin{theorem}
Если $\tau$ - марковский момент, то $\{\tau < t\} \in \mathcal{F}_t, \forall t \in T$. 
\end{theorem}
\begin{proof}[Идея доказательства]
Разбираем два случая: $t_* \in T$, $t_* \notin T$. В каждом из них граничные случаи следуют из того, что $\emptyset \in \mathcal{F}_t$, а остальное доказывается взятием последовательности $s_n \uparrow u = \sup\{s \in T \mid s < t\}$. В ней конечное число различных чисел, дальше разбираем случаи $u \notin T$ или $u = t$ и $t > u \in T$. 
\end{proof}
\begin{theorem}
Пусть множество $T$ - счётное. Тогда функция $\tau$ является марковским моментом $\Longleftrightarrow \{\tau = t\} \in \mathcal{F}, \forall t \in T$.
\end{theorem}
\begin{proof}[Идея доказательства]
Очевидно из предыдущей теоремы и того, что $\{\tau = t\} = \{\tau \le t\} \backslash \{\tau < t\}$.
\end{proof}
\begin{theorem}
Функция $\tau$ является марковским моментом относительно непрерывной справа фильтрации с выпуклым множеством $T$ $\Longleftrightarrow \{\tau < t\} \in \mathcal{F}_t, \forall t \in T$ и $\{\tau = \infty\} \in \mathcal{F}_{t^*}$, если $t^* \in T$.
\end{theorem}
\begin{proof}[Идея доказательства]
Следствие $\Longrightarrow$ очевидно из второй теоремы и $\{\tau = \infty\} = \Omega \backslash \{\tau \le t^*\}$. \newline Следствие $\Longleftarrow$ доказывается обратным рассуждением для $t^*$ и представлением $\{\tau \le t\} = \cap_{n \ge \frac{1}{(s - t)}}\{\tau < t + \frac{1}{n}\}$.
\end{proof}
\begin{theorem}
Пусть $\tau, \sigma: \Omega \rightarrow T \cup \{\infty\}$. Если $\tau$ - марковский момент относительно расширенной фильтрации $\mathrm{F}_t$ и почти всюду $\tau = \sigma$, то $\sigma$ - марковский момент относительно $\mathrm{F}_t$.
\end{theorem}
\begin{proof}[Идея доказательства]
Вводим $A = \{\tau = \sigma\}$, тогда $P(A^c) = 0$, оба множества лежат в $\mathcal{F}_t$ для любого $t \in T$. Представим $\{\sigma \le t\} = (\{\tau \le t\} \cap A) \cup (\{\sigma \le t\} \cap A^c)$.
\end{proof}
\begin{theorem}
Если $\tau$ - марковский момент, то функция $\tau \wedge t$ измерима относительно $\mathcal{F}_t$ для любого $t \in T$.
\end{theorem}
\begin{proof}[Идея доказательства]
Достаточно доказать, что $\{(\tau \wedge t) \le u\} \in \mathcal{F}_t, \forall u \in R$. Если $t < u$, то, очевидно $\Omega \in \mathcal{F}_t$. Иначе рассуждаем точно как при доказательстве второй теоремы.
\end{proof}
\begin{theorem}
Если $\tau, \sigma$ - марковские моменты, то $\tau \wedge \sigma, \tau \vee \sigma$ - тоже марковские моменты.
\end{theorem}
\begin{proof}[Идея доказательства]
$\{(\tau \wedge \sigma) \le t\} = \{\tau \le t\} \cup \{\sigma \le t\}$, $\{(\tau \vee \sigma) \le t\} = \{\tau \le t\} \cap \{\sigma \le t\}$.
\end{proof}

\section{Сигма-алгебры, связанные с марковскими моментами (3.6.8 - 3.6.9)}
\begin{definition}
С каждым $\mathrm{F}_T$-марковским моментом $\tau$ связаны две $\sigma$-алгебры $\mathcal{F}_\tau$ и $\mathcal{F}_{\tau-}$. \newline
Сигма-алгебра $\mathcal{F}_\tau$ состоит из множеств $A \in \sigma(\mathcal{F}_t, t \in T)$, для которых $A \cap \{\tau \le t\} \in \mathcal{F}_t$. Она называется \textit{сигма-алгеброй событий, предшествующих марковскому моменту $\tau$}. \newline
Сигма-алгебра $\mathcal{F}_{\tau-}$ порождается множествами $A \cap \{t < \tau\}, A \in \mathcal{F}_t$. Если дополнительно $t_* \in T$, то к классу порождающих множеств добавляются ещё все множества $A \in \mathcal{F}_{t_*}$.
\end{definition}
\begin{theorem}
Пусть $\tau, \sigma$ - марковские моменты. Тогда верны следующие утверждения:
\begin{enumerate}
    \item $\mathcal{F}_{\tau-} \subseteq \mathcal{F}_\tau$.
    \item $\mathcal{F}_{\tau} \subseteq \mathcal{F}_\sigma, \mathcal{F}_{\tau-} \subseteq \mathcal{F}_{\sigma-}$ при $\tau \le \sigma$.
    \item $\mathcal{F}_{\tau \wedge \sigma} = \mathcal{F}_\tau \cap \mathcal{F}_\sigma$.
    \item $\mathcal{F}_{\tau} \subseteq \mathcal{F}_{\sigma-}$ при $\tau < \sigma$.
    \item $\{\tau \le \sigma\}, \{\tau = \sigma\} \in \mathcal{F}_{\tau \wedge \sigma}$ \newline $\{\tau < \sigma\} \in \mathcal{F}_\tau \cap \mathcal{F}_{\sigma-}$.
    \item Если $A \in \mathcal{F}_\tau$, то $A \cap \{\tau \le \sigma\}, A \cap \{\tau = \sigma\} \in \mathcal{F}_\sigma$.
    \item Если $A \in \mathcal{F}_\tau$, то $A \cap \{\tau < \sigma\} \in \mathcal{F}_\tau \cap \mathcal{F}_{\sigma-}$.
\end{enumerate}
\end{theorem}
\begin{proof}[Идея доказательства] Доказываем пункты по очереди, но многие делаются тупо руками, без идеи.
\begin{enumerate}
    \item Показываем, что все порождающие множества левой части лежат в правой.
    \item Опять то же самое.
    \item В силу 2 верно включение $\mathcal{F}_{\tau \wedge \sigma} \subseteq \mathcal{F}_\tau \cap \mathcal{F}_\sigma$. Доказываем обратное.
    \item Берём любое счётное всюду плотное $S' \subseteq T$. Добавляем к нему лежащие в $T$ точные грани. Вводим $S = S' \cup (Q \cap T)$. Дальше всё та же привычная рутина: для любого множества $A \in \mathcal{F}_\tau$ доказываем, что оно в $\mathcal{F}_{\sigma-}$, представляя его как $A = \cup_{s \in S}(A \cap \{\tau \le s\}) \cap \{s < \sigma\}$.
    \item По теореме 6 предыдущего билета доказываем, что $\{\tau \le \sigma\} \in \mathcal{F}_\sigma, \{\tau \le \sigma\} \in \mathcal{F}_\tau$, значит $\{\tau \le \sigma\} \in \mathcal{F}_\tau \cap \mathcal{F}_\sigma = \mathcal{F}_{\tau \wedge \sigma}$. Теми же рассуждениями можно показать то же самое для $\{\sigma \le \tau\}$. Тогда $\{\tau = \sigma\}, \{\tau < \sigma\}$ представимы по очереди через имеющиеся. Остаётся доказать, что $\{\tau < \sigma\} \in \mathcal{F}_{\sigma-}$. Доказываем через $S$ из предыдущего пункта похожим образом.
    \item Используем предыдущий пункт, а дальше руками... Никакой идеи выделить не получается.
    \item Объединяем 4 и 5 пункты.
\end{enumerate}
\end{proof}

\section{Измеримость марковских моментов и другие из свойства (3.6.10 - 3.6.15)}
\begin{theorem}
Любой марковский момент $\tau$ измерим относительно $\sigma$-алгебры $\mathcal{F}_{\tau-}$.
\end{theorem}
\begin{proof}[Идея доказательства]
Достаточно доказать, что $\{u < \tau\} \in \mathcal{F}_{\tau-}$. Предполагаем, что $t_* \in T$. Разберём четыре случая: $u < t_*, u = t_*$ и $u > t_*$ при $u \in T$ и $u \notin T$. Первые три очевидны, в последнем берём $v = \sup\{s \in T \mid s < u\}$. Если $v \in T$, то очевидно, иначе $\exists \{s_n\}, s_n \in T, s_n \uparrow v$. В ней может быть конечное число различных чисел. По доказанному в предыдущем билете множество $\{\tau \le s_n\} \in \mathcal{F}_{\tau-}$, тогда $\{\tau \le u\} \in \mathcal{F}_{\tau-}$.
\end{proof}
\begin{theorem}
Пусть дан марковский момент $\tau$ и $\mathcal{F}_\tau$-измеримая функция $\sigma: \Omega \rightarrow T \cup \{\infty\}$. Тогда:
\begin{enumerate}
    \item Если $\tau \le \sigma$, то $\sigma$ - марковский момент.
    \item $\forall A \in \mathcal{F}_\tau$ функция $\tau_A = \tau\mathbf{1}_A + \infty\mathbf{1}_{A^c}$ является марковским моментом.
\end{enumerate}
\end{theorem}
\begin{proof}[Идея доказательства]
\begin{enumerate}
    \item $\{\sigma \le t\} = \{\sigma < t\} \cap \{\tau < t\} \in \mathcal{F}_t$, т.к. $\{\sigma < t\} \in \mathcal{F}_\tau$.
    \item Функция $\mathcal{F}_\tau$-измерима и $\tau \le \tau_A$.
\end{enumerate}
\end{proof}
\begin{theorem}
Для любого марковского момента $\tau$ относительно фильтрации $\mathrm{F}_T$ с выпуклым множеством $T$ существуют такие марковские моменты $\tau_n$, что каждый из них принимает конечное число значений и $\tau_n \downarrow \tau$.
\end{theorem}
\begin{proof}[Идея доказательства]
Возьмём $T \ni a_n \downarrow t_*, T \ni b_n \uparrow t^*$. Если какая-то из граней лежит в $T$, то всю соответствующую последовательность берём равной этой грани. \newline Разбиваем $a_n = t_{n, 0} < \ldots < t_{n, m_n} = b_n$ так, чтобы с ростом $n \rightarrow \infty$ старые точки были подмножествами новых и расстояния между ними стремились к $0$. Определим $\vartheta_n(t)$ на $T \cup [t^*, \infty]$, положив $\vartheta_n(t) = a_n, t \in T, t \le a_n; \vartheta_n(t) = t_{n,k}, t \in (t_{n, k-1}, t_{n, k}]; \vartheta_n(t) = \infty, t > b_n$. Эта функция возрастает и всегда $\vartheta_n(t) \ge t$. Берём $\tau_n = \vartheta_n(\tau)$.
\end{proof}
\begin{theorem}
Пусть дана последовательность марковских моментов $\{\tau_n\}$. Если она убывает и $\forall \omega \in \Omega$ найдётся номер $m$, начиная с которого все $\tau_n(\omega) = \tau_m(\omega)$, то функция $\tau = \lim_{n \rightarrow \infty}\tau_n$ является марковским моментом.
\end{theorem}
\begin{proof}[Идея доказательства]
$\{\tau \le t\} \subseteq \cap_{n = 1}^\infty \{\tau_n \le t\}$. Но для любого $\omega \in \cap_{n = 1}^\infty \{\tau_n \le t\}$ найдётся номер $m : \tau(\omega) = \tau_m(\omega) \le t$. Тогда $\{\tau \le t\} = \cap_{n = 1}^\infty \{\tau_n \le t\} \in \mathcal{F}_t$.
\end{proof}
\begin{theorem}
Пусть даны марковские моменты $\tau_n$. Если $\forall t_n \in T \sup_{n \rightarrow \infty}t_n \in T \cup \{\infty\}$, то $\tau = \sup_{n \rightarrow \infty}\tau_n$ - марковский момент и $\cup_{n = 1}^\infty \mathcal{F}_{\tau_n} \subseteq \mathcal{F}_\tau$.
\end{theorem}
\begin{proof}[Идея доказательства]
Тот факт, что это марковский момент, доказывается по той же идее, что и в предыдущей теореме. Второе утверждение следует из $\tau_n \le \tau$ и теореме из предыдущего билета.
\end{proof}
\begin{theorem}
Пусть даны марковские моменты $\tau_n$ относительно фильтрации $\mathrm{F}_T$. Если фильтрация непрерывна справа, а $T = \{1, 2, \ldots\}$ или $T = [a, b]$, или $T = [a, \infty)$, то функции $$\lim\sup \tau_n, \lim\inf \tau_n, \sup \tau_n, \inf \tau_n, n \rightarrow \infty$$ являются марковскими моментами и $\mathcal{F}_{\inf \tau_n} = \cap_{n = 1}^\infty \mathcal{F}_{\tau_n}$.
\end{theorem}
\begin{proof}[Идея доказательства]
Доказываем в два этапа:
\begin{enumerate}
    \item Предположим, что $T = \{1, 2, \ldots\}$. По предыдущей теореме супремум является марковским моментом, инфимум тоже, поскольку $\{\tau \le t\} = \cup_{n = 1}^\infty\{\tau_n \le t\}$. Пределы представимы через инфимум и супремум. Равенство для сигма-алгебр выводится из теоремы из предыдущего билета.
    \item При $T = [a, b]$ или $T = [a, \infty]$ опять говорим, что супремум и инфимум - марковские моменты. И опять тогда все четыре функции - марковские моменты. Заметим, что $\mathcal{F}_\tau \subseteq \cap_{n = 1}^\infty \mathcal{F}_{\tau_n}$. Берём $A \in \cap_{n = 1}^\infty \mathcal{F}_{\tau_n}$. По теореме из предыдущего билета и из непрерывности $\mathrm{F}_T$ показывается, что $A \in \mathcal{F}_\tau$. Тогда $\cap_{n = 1}^\infty \mathcal{F}_{\tau_n} \subseteq \mathcal{F}_\tau$. Вложение в обе стороны обуславливает равенство.
\end{enumerate}
\end{proof}

\section{Предсказуемые марковские моменты (3.7.1 - 3.7.7)}
Теперь у нас $T = [0, \infty)$ и фильтрация $\mathrm{F} = \{\mathcal{F}_t \mid \mathcal{F}_t \subseteq \mathcal{F}, t \ge 0\}$. Соответственно, все марковские моменты берутся относительно этой фильтрации, теперь это функции вида $\tau: \Omega \rightarrow \overline{R}_+$.
\begin{definition}
Марковский момент называется \textit{предсказуемым}, если $\exists \tau_n\uparrow, \tau_n < \tau$ на множестве $\{\tau > 0\}$ такая, что $\lim_{n \rightarrow \infty} \tau_n = \tau$. Сама последовательность называется \textit{предвещающей}.
\end{definition}
Дальше идут три примера:
\begin{enumerate}
    \item $\forall \alpha \ge 0, A \in \mathcal{F}_\alpha: \tau = \alpha\mathbf{1}_A + \infty\mathbf{1}_{A^c}$ - предсказуемый марковский момент. Марковский момент - потому что теорема в предыдущем билете, предсказуемый - потому что если $\alpha > 0$, то есть $\alpha_n \uparrow$ такая, что $\alpha_n \rightarrow \alpha$, ну и предвещающей последовательностью берём $\{\tau_n \wedge n\}$, где $\tau_n = \alpha_n\mathbf{1}_A + \infty\mathbf{1}_{A^c}$. А если $\alpha = 0$, то возьмём $\{\tau_A \wedge n\}$.
    \item Сумма $\tau + \sigma$, где $\tau$ - марковский момент, а $\sigma$ - предсказуемый марковский момент, является предсказуемым марковским моментом. В качестве предвещающей последовательности берём $\{\tau + \sigma_n\}$, где $\sigma_n$ - предвещающая последовательность для $\sigma$. Каждая из таких сумм - марковский момент, потому что была такая задача в предыдущем параграфе, причём без решения, так что, видимо, просто потому что.
    \item $X$ - случайный процесс: $\mathrm{F}$-согласованный, непрерывный, вещественный (нашедший своё место в жизни, самореализовавшийся, семья, дети, все дела. Кто-нибудь помнит, что эти слова вообще значат?). Тогда $\forall c \in R: \tau = \inf\{t \ge 0 \mid X_t \ge c\}$ - предсказуемый марковский момент. Потому что, бл*ть, гладиолус, дальше на страницу чё-то расписано, но сводится к тому, что предвещающая последовательность - $\{\tau_n \wedge n\}$, где $\tau_n = \inf\{t \ge 0 \mid X_t \ge c - \frac{1}{n}\}$. 
\end{enumerate}
Настало время теорем. Особенно неожиданным результатом кажется первое утверждение следующей теоремы.
\begin{theorem}
Предсказуемый марковский момент является марковским моментом; $\mathcal{F}_{\tau-} = \sigma(\cup_{n = 1}^\infty \mathcal{F}_{\tau_n})$ для любой предвещающей последовательности $\tau_n$.
\end{theorem}
\begin{proof}[Идея доказательства]
Мне кажется, что первое утверждение начисто лишено смысла. Доказывается оно в строку, но лучше обсудить, что имеется в виду. \newline Для доказательства второго утверждения опять показываем два вложения. Для доказательства $\subseteq$ показываем, что множества, порождающие $\mathcal{F}_{\tau-}$, лежат в $\sigma(\cup_{n = 1}^\infty \mathcal{F}_{\tau_n})$, для обратного вложения достаточно показать, что $\forall n: \mathcal{F}_{\tau_n} \subseteq \mathcal{F}_{\tau-}$.
\end{proof}
\begin{theorem}
Пусть $\tau, \sigma$ - предсказуемые марковские моменты. Тогда:
\begin{enumerate}
    \item $\tau \wedge \sigma, \tau \vee \sigma$ - предсказуемые марковские моменты.
    \item $\{\tau \le \sigma\}, \{\tau < \sigma\}, \{\tau \ge \sigma\}, \{\tau > \sigma\}, \{\tau = \sigma\} \in \mathcal{F}_{\tau-}\cap\mathcal{F}_{\sigma-}$.
    \item $A \cap \{\tau \le \sigma\}, A \cap \{\tau < \sigma\}, A \cap \{\tau = \sigma\} \in \mathcal{F}_{\tau-}\cap\mathcal{F}_{\sigma-}$ для любого $A \in \mathcal{F}_{\tau-}$
\end{enumerate}
\end{theorem}
\begin{proof}[Идея доказательства]
\begin{enumerate}
    \item Очевидно, что $\tau_n \wedge \sigma_n, \tau_n \vee \sigma_n$ будут предвещающими последовательностями.
    \item Доказываем, что $\{\tau \le \omega\} = \cap_{n = 1}^\infty \cup_{m = 1}^\infty \{\tau_n \le \sigma_m\}$. Тогда $\{\tau_n \le \sigma_m\} \in \mathcal{F}_{\tau_n} \cap \mathcal{F}_{\omega_m} \subseteq \mathcal{F}_{\tau-} \cap \mathcal{F}_{\omega-}$ в силу теорем, доказанных в предыдущих билетах. Тогда $\{\tau \le \sigma\} \in \mathcal{F}_{\tau-} \cap \mathcal{F}_{\omega-}$, можем ещё в рассуждениях поменять местами $\tau$ и $\sigma$, а ещё представить равенство как пересечение и строгое неравенство как разность с равенством.
    \item По предыдущему пункту принадлежность указанным множеств классу $\mathcal{F}_{\tau-}$ установлена. Осталось показать принадлежность $\mathcal{F}_{\sigma-}$. Берём равенство с пересечением объединений из предыдущего пункта, пересекаем с любым $A \in \Omega$, применяем теорему из 14-го билета учитывая, что если $A \in \cup_{r = 1}^\infty \mathcal{F}_{\tau_r}$, то $A \in \mathcal{F}_{\tau_r}$ для некоторого $r$. Получаем, что $A \cap \{\tau \le \sigma\} \in \mathcal{F}_{\sigma-}$. Обозначив $\mathcal{L}$ класс множеств из $\mathcal{F}_{\tau-}$, для которых выполнено $A \cap \{\tau \le \sigma\} \in \mathcal{F}_{\sigma-}$, показываем, что этот класс является сигма-алгеброй. Откуда вытекает то, что нужно. Остальные множества выводятся из данного.
\end{enumerate}
\end{proof}
\begin{theorem}
Пусть даны предсказуемые марковские моменты $\tau_n$.
\begin{enumerate}
    \item Если $\tau_n \uparrow$, то $\tau = \lim_{n \rightarrow \infty}\tau_n$ - предсказуемый марковский момент.
    \item Если $\tau_n \downarrow$ и для любого $\omega \in \Omega$ найдётся номер $m$, начиная с которого $\tau_n(\omega) = \tau_m(\omega)$, то $\tau = \lim_{n \rightarrow \infty}\tau_n$ - предсказуемый марковский момент.
\end{enumerate}
\end{theorem}
\begin{proof}[Идея доказательства]
\begin{enumerate}
    \item Пусть $\{\tau_{n,m}\}$ предвещает $\tau_n$. Обозначим $\sigma_m = \max\{\tau_{i, m}; i = 1,\ldots, m\}$. По доказанным в предыдущих билетах теоремам эта функция, а также предел в условии теоремы являются марковскими моментами. Далее на страницу доказываем, что $\{\sigma_m\}$ предвещает $\tau$.
    \item Ещё на страницу доказываем, что на сей раз $\sigma_m = \min\{\tau_{i, m}; i = 1, \ldots, m\}$ является предвещающей для $\tau$.
\end{enumerate}
\end{proof}
\begin{theorem}
Пусть $\tau$ - предсказуемый марковский момент. Тогда $\forall A \in \mathcal{F}_{\tau-}$ функция $\tau_A = \tau\mathbf{1}_A + \infty\mathbf{1}_{A^c}$ - предсказуемый марковский момент.
\end{theorem}
\begin{proof}[Идея доказательства]
Обозначим $\mathcal{L}$ класс множеств $A \in \mathcal{F}_{\tau-}$, для которых $\tau_A, \tau_{A^c}$ являются марковскими моментами. Пусть порождающая последовательность для $\tau$ - $\{\tau_k\}$. Тогда сигма-алгебра $\mathcal{F}_{\tau-}$ порождается алгеброй $\mathcal{A} = \cup_{k = 1}^\infty \mathcal{F}_{\tau_k}$. Доказываем, что $\mathcal{A} \subseteq \mathcal{L}$ (то есть, что $\forall A \in \mathcal{A}: \tau_A, \tau_{A^c}$ - предсказуемые марковские моменты). Далее доказываем, что $\mathcal{L}$ - $\lambda$-класс. После чего по теореме Серпинского (о, чё вспомнили, это первый билет, если что!) $\mathcal{F}_{\tau-} = \sigma(\mathcal{A}) \subseteq \mathcal{F}$, значит доказали вложение в обе стороны, значит $\mathcal{L} = \mathcal{F}_{\tau-}$.
\end{proof}
\end{document}
